\begin{frame}
    \frametitle{Research Objectives}
    \begin{block}{Technical Gap}
      \begin{itemize}
        \item Previous efforts towards nuclear reactor optimization focuses
        on optimizing classical reactor parameters such as radius of fuel pellet 
        and clad, enrichment of fuel, pin pitch, etc.
        \item Additive manufacturing advancements enable reactor designers to 
        optimize beyond conventional manufacturing shapes that are easy to 
        manufacture such as slabs as fuel planks, cylinders as rods, axis-aligned
        coolant channels 
        \item Applying evolutionary algorithms to nuclear design problems is not new,
        however, evolutionary algorithm setup is highly customizable, A reactor 
        designer unfamiliar with evolutionary algorithms will have to go through 
        the cumbersome process of customizing a genetic algorithm for their
        needs and determine which operators and hyperparameters work best for their problem
      \end{itemize}
    \end{block}
    \begin{block}{Scope 2: }
      \begin{itemize}
        \item f
        \end{itemize}
    \end{block}
  \end{frame}