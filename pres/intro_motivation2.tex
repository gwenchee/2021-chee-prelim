\begin{frame}
    \frametitle{Motivation}
    \begin{block}{Impact of Additive Manufacturing Technology Advancements on 
        Reactor Design Optimization}
        \begin{itemize}
            \item Additive manufacturing technology, popularly known as `3D printing', 
            has advanced and altered the manufacturing and design of engineered hardware
            \cite{simpson_considerations_2019}. 
            \item With further advancement of additive manufacturing technologies, a reactor 
            core could be 3D printed in the near future. 
            \acrlong{ORNL} leads this initiative through the 2019 \acrlong{TCR} Demonstration 
            Program. 
            \item Leveraging additive manufacturing technology enables us to surpass classical 
            manufacturing constraints, such as straight fuel channels or homogenous fuel 
            enrichment, and optimize for arbitrary geometries and parameters 
            such as non-uniform channel shapes, and inhomogeneous fuel distribution 
            throughout the core. 
          \end{itemize}
    \end{block}
  \end{frame}

  \begin{frame}
    \frametitle{Motivation}
    \begin{block}{Evolutionary Algorithms for Reactor Design Optimization}
        \begin{itemize}
            \item Evolutionary algorithms have proven successful in optimizing multi-objective 
            problems such as reactor design optimization, as they can find solutions at the global 
            optimum and also take advantage of parallel systems. 
            \item Evolutionary algorithms imitate natural selection to evolve solutions 
            \cite{renner_genetic_2003}:
            \begin{itemize}
                \item maintaining a population of solutions
                \item allowing more optimal solutions to reproduce 
                \item letting less optimal solutions to die off resulting in 
                solutions that are better than previous generations
            \end{itemize}
            \end{itemize}
    \end{block}
  \end{frame}
    