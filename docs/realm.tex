\chapter{REALM}
% Main Gist 
% - Genetic Algorithms / DEAP have been applied to a multitude of problems, 
%   but barely to nuclear. DEAP was created to be easily coupled with other
%   codes and to flexibly build your own reactor. 
% Structure 
% - Realm framework
% - Coupling to OpenMC and Moltres

In this chapter, we introduce the \gls{REALM} Python framework developed for this
proposed work.
\gls{REALM} is an optimization tool that applies \gls{EA} optimization to 
nuclear reactor design. 
The tool is:  
\begin{itemize}
    \item Open-source: We utilized a well-documented open-source \gls{EA} Python 
    package to drive the genetic algorithm optimization. We utilized established 
    open-source nuclear transport, OpenMC \cite{romano_openmc_2013}, and 
    thermal-hydraulics, Moltres \cite{lindsay_introduction_2018}, software to 
    compute the objective function and constraints. We also provide a simple 
    tutorial for future developers to follow for coupling other nuclear software 
    to the \gls{REALM} package.  
    \item Flexible: Our goal is to utilize \gls{REALM} to explore arbitrary 
    reactor geometries and inhomogeneous fuel distributions. However, we 
    acknowledge that future users might want to utilize \gls{REALM} other
    arbitrary parameters that we overlooked. Thus, we designed the \gls{REALM}
    framework with this in mind. The user is able to vary any imaginable parameter 
    as \gls{REALM} edits the neutronics/thermal-hydraulics software's input file 
    with a templating method.
    \item Usable: 
    \item Parallel: \gls{REALM} runs parallel on \gls{HPC} machines using the 
    \texttt{multiprocessing\_on\_dill} Python package 
    \cite{smallshire_multiprocessing_on_dill_nodate}.
    \item Reproducible: 
    \item Effective: 
    \item Usable: 
\end{itemize}