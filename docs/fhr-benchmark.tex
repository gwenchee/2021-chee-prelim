\chapter{Fluoride Salt Cooled High Temperature Reactor Benchmark}
\label{chap:fhr-benchmark}
% Main Gist 
% - The work I did for the FHR Benchmark
% Structure 
% - Specifications of benchmark problem
% - Results from benchmark
% Appendix? 
% - More details about the geometry 

In 2019, \gls{OECD}-\gls{NEA} and \gls{Georgia Tech} initiated the
\gls{FHR} benchmark \cite{noauthor_fluoride_nodate}. 
\glspl{FHR} are cooled with liquid molten salt and fuelled with 
\gls{TRISO}-based fuel. 
The \gls{FHR} studied in this benchmark has plate-based fuel in hexagonal fuel 
elements, i.e. \gls{AHTR} design. 
The benchmark has several phases, starting with single fuel element 
simulation without burnup, and gradually extending to full core depletion. 
Table \ref{tab:phases} outlines the complete and incomplete benchmark phases.

\begin{table}[H]
    \centering
    \onehalfspacing
    \caption{Phases of the \gls{FHR} benchmark \cite{noauthor_fluoride_nodate}.}
	\label{tab:phases}
    \footnotesize
    \begin{tabular}{lclc}
    \hline 
    \textbf{Phases}& \textbf{Sub-phases} & \textbf{Description} & \textbf{Completed?} \\
    \hline
    \multirow{ 3}{5cm}{\textbf{Phase I: Fuel assembly (2D/3D with depletion)}} & I-A & 2D model, steady state (no depletion) & \checkmark\\
    &I-B & 2D model depletion & \checkmark\\
    &I-C & 3D model depletion &\\
    \hline
    \multirow{2}{5cm}{\textbf{Phase II: 3D full core with depletion}}&II-A & Steady-state (no depletion) &\\
    &II-B & Depletion &\\
    \hline 
    \multirow{ 2}{5.5cm}{\textbf{Phase III: 3D full core with feedback \& multicycle analysis}}&III-A & Full core depletion with feedback &\\
    &III-B & Multicycle analysis &\\
    \hline
    \end{tabular}
\end{table}

In the subsequent sections, we will describe the benchmark specifications, 
phases I-A and I-B, and their results. 

\section{Benchmark Specifications}
The \gls{AHTR} has 3400 MWt thermal power and 1400 MW electric power 
\cite{varma_ahtr_2012}. 
Figure \ref{fig:reactor-schematic} shows the reactor schematic and a vertical 
cut of the reactor vessel. 
Figure \ref{fig:ahtr} shows the core configuration and fuel element. 
\begin{figure}[]
    \centering
    \includegraphics[width=0.9\linewidth]{reactor-schematic.png} 
    \caption{Reactor schematic (left) and vessel (right) 
    \cite{noauthor_fluoride_nodate}.}
    \label{fig:reactor-schematic}
\end{figure}
The prismatic \gls{AHTR}'s fuel element is shown in Figure 
\ref{fig:ahtr-fuel-element}.  
It features plate-type fuel with hexagonal fuel assembly consisting of eighteen 
plates arranged in three diamond-shaped sectors, with a central Y-shaped 
structure and external channel (wrapper).
The diamond-shaped sections have 120-deg rotational symmetry with each other 
\cite{varma_ahtr_2012,ramey_monte_2018,noauthor_fluoride_nodate}. 
\begin{figure}[]
    \centering
    \includegraphics[width=0.5\linewidth]{ahtr-fuel-element.png} 
    \caption{FHR fuel element.}
    \label{fig:ahtr-fuel-element}
\end{figure}

The external channel wrapper and structural Y-shape as seen in Figure 
\ref{fig:y-shape} are made of C-C composite and have extra notches for the 
fuel plates to slide in. 
The gap between the fuel elements and fuel plates are filled with \gls{FLiBe}
coolant. 
At the center of the Y-shape structure is the Y-shaped control blade slot, 
which contains \gls{FLiBe} coolant when the control blade is not in the slot
\cite{varma_ahtr_2012,ramey_monte_2018,noauthor_fluoride_nodate}.
\begin{figure}[]
    \centering
    \includegraphics[width=0.8\linewidth]{y-shape.png} 
    \caption{FHR fuel element's structural components \cite{noauthor_fluoride_nodate}.}
    \label{fig:y-shape}
\end{figure}
Each fuel plank is made of an isostatically pressed carbon with fuel stripes 
on each outer side of the plank as seen in Figure \ref{fig:ahtr-fuel-plank}. 
\begin{figure}[]
    \centering
    \includegraphics[width=0.8\linewidth]{ahtr-fuel-plank.png} 
    \caption{2D view of FHR fuel plank.}
    \label{fig:ahtr-fuel-plank}
\end{figure}
The fuel stripes are prismatic regions composed of a graphite matrix filled with 
a cubic lattice of \gls{TRISO} particles with a 40\% packing fraction. 
The lattice is 210 \gls{TRISO} particles wide in x-direction, 4 particles deep in 
the y-direction, and 5936 particles tall in the z-direction. 
Each \gls{TRISO} particle has 5 layers: Oxycarbide fuel kernel, porous carbon 
buffer, inner pyrolytic carbon, silicon carbide layer, and the outer pyrolitic 
carbon as seen in Figure \ref{fig:ahtr-triso}
\begin{figure}[]
    \centering
    \includegraphics[width=0.4\linewidth]{ahtr-triso.png} 
    \caption{TRISO particle schematic \cite{noauthor_fluoride_nodate}.}
    \label{fig:ahtr-triso}
\end{figure}

\section{Phase I}
\label{sec:phase1}
Phases I consists of a steady-state 2D model of one \gls{FHR} fuel element. 
For a single fuel element, the internal 120-degree rotational symmetry is 
represented by periodic boundary conditions as seen in Figure \ref{fig:bc}. 
\begin{figure}[]
    \centering
    \includegraphics[width=0.8\linewidth]{bc.png} 
    \caption{Visualization of periodic boundary conditions for a single fuel 
    element \cite{noauthor_fluoride_nodate}.}
    \label{fig:bc}
\end{figure}
The benchmark required the following results for phases I-A and I-B:
\begin{enumerate}[label=(\alph*)]
    \item Effective multiplication factor 
    \item Reactivity coefficients ($\beta_{eff}$, fuel Doppler coefficient, FLiBe 
    temperature coefficient, graphite temperature coefficient)
    \item Tabulated fission source distribution, at several levels of granularity 
    (by fuel plate, by fuel stripe, by 1/5-th fuel stripe). Optional: visualized fission 
    density distribution.
    \item Neutron flux averaged over the whole model tabulated in 3 coarse energy groups. 
    \item Visualized distribution of the neutron flux distribution, in 3 coarse energy groups
    \item Neutron spectrum, fuel assembly average. Optional: by region.
\end{enumerate}

The following subsections will report the equations we used to calculate the
results listed above. 

\subsubsection{Reactivity Coefficients (b)}
Effective delayed neutron fraction ($\beta_{eff}$) is the fraction of delayed 
neutrons in the core at creation at thermal energies. 
We assumed 1 energy group and 6 delayed groups for $\beta_{eff}$: 
\begin{align}
    \beta_{eff} = \sum_k \beta_k
\end{align}

We calculated each reactivity coefficient and its corresponding uncertainty 
with these equations: 
\begin{align}
    \frac{\Delta \rho}{\Delta T} &= 
    \frac{\rho_{T_{high}}-\rho_{T_{low}}}{T_{high}-T_{low}} [\frac{pcm}{K}] \\
    \delta \frac{\Delta \rho}{\Delta T} &= 
    \frac{\sqrt{\delta (\rho_{T_{high}})^2+(\delta \rho_{T_{low}})^2}}{T_{high}-T_{low}} [\frac{pcm}{K}] 
\end{align}

\subsubsection{Fission Source Distribution / Fission Density (c)}
We calculated \gls{FD} with OpenMC's \texttt{fission} score (f) for a region 
divided by the average \texttt{fission} score of all the regions. 

\begin{align*}
    FD_i &=  \frac{f_i}{f_{ave}}
    \intertext{where:}
    f_i &= \mbox{Total fission reaction rate [reactions/src]} \\
    f_{ave} &= \mbox{average of all $f_i$ [reactions/src]}
\end{align*}
The uncertainty calculations for $f_{ave}$ and $FD_i$: 
\begin{align*}
    \delta f_{ave} &= \frac{1}{N}\sqrt{\sum_i^Nf_i^2} \\
    \delta FD_i &= |FD_i| \sqrt{(\frac{\delta f_i}{f_i})^2+(\frac{\delta f_{ave}}{f_{ave}})^2}
    \intertext{where:}
    N &= \mbox{No. of fission score values} 
\end{align*}

\subsubsection{Neutron Flux (d, e, f)}
OpenMC's \texttt{flux} score are in [$\frac{n * cm}{src}$] units. 
For the benchmark, we converted flux to [$\frac{n}{cm^2s}$] units
using the following equations:  

\begin{align}
    \Phi_c &= \frac{N* \Phi_o}{V} 
    \intertext{where:} 
    N &= \frac{P*\nu}{Q*k} \nonumber \\
    \Phi_c &= \mbox{Converted Flux [$\frac{neutrons}{cm^2s}$]} \nonumber \\ 
    \Phi_o &= \mbox{Original Flux [$\frac{neutrons* cm}{src}$]} \nonumber \\
    N &= \mbox{Normalization factor [$\frac{src}{s}$]} \nonumber \\
    V &= \mbox{Volume of fuel assembly [$cm^3$]} \nonumber \\
    P &= \mbox{Power [$\frac{J}{s}$]} \nonumber \\
    \nu &= \mbox{$\frac{\nu_f}{f}$ [$\frac{neutrons}{fission}$]} \nonumber \\
    Q &= \mbox{Energy produced per fission [$\frac{J}{fission}$]} = \mbox{$3.2044*10^{-11}$ J per $U_{235}$ fission} \nonumber \\
    k &= \mbox{$k_{eff}$ [$\frac{neutrons}{src}$]} \nonumber 
\end{align}
Flux standard deviation: 
\begin{align}
    \delta \Phi_c = \Phi_c * 
    \sqrt{(\frac{\delta \Phi_o}{\Phi_o})^2+ (\frac{\delta \nu_f}{\nu_f})^2 
    + (\frac{\delta k}{k})^2 + (\frac{\delta f}{f})^2}
\end{align}
We calculated reactor power based on the given reference specific power 
($P_{sp}$) of 200 $\frac{W}{gU}$. 
\begin{align}
    P &= P_{sp} * V_F * \rho_F * \frac{wt\%_{U}}{100} 
    \intertext{where:}
    V_F &= \mbox{Volume of fuel [$cm^3$]} = \frac{4}{3} \pi r_1^3 * 101 * 210 * 4 * 2 * 6 * 3 \nonumber \\
    \rho_F &= \mbox{density of fuel [$g/cc$]} \nonumber \\
    wt\%_{U} &= \frac{at\%_{U235} * AM_{U235} + at\%_{U238} * AM_{U238}}{\sum (at\%_i * AM_i)} * 100 \nonumber\\
    AM &= \mbox{atomic mass} \nonumber
\end{align}

\subsection{Phase I-A}
For phase I-A, the benchmark specifies analysis for 9 cases for a 2D model of 
one fresh fuel element. 
Table \ref{tab:phase1a-cases} outlines each of the cases. 
\begin{table}[H]
    \centering
    \onehalfspacing
    \caption{Description of cases in Phase I-A of the \gls{FHR} benchmark \cite{noauthor_fluoride_nodate}.}
	\label{tab:phase1a-cases}
    \footnotesize
    \begin{tabular}{p{0.05\textwidth}|p{0.9\textwidth}}
    \hline 
    \textbf{Case} & \textbf{Description} \\
    \hline
    1A & Reference case. Hot full power (HFP), with temperatures of 1110K for 
    fuel kernel and 948K for coolant and all other materials (including TRISO 
    particle layers other than fuel kernel). Nominal (cold) dimensions, 
    9 wt\% enrichment, no \gls{BP}, \glspl{CR} out.\\
    \hline
    2AH & \gls{HZP} with uniform temperature of 948 K, 
    otherwise same as CASE 1A. Comparison with CASE 1A provides HZP-to-HFP power 
    defect.\\
    \hline 
    2AC & \gls{CZP}. Same as CASE 2AH, but with uniform temperature 
    of 773 K. Comparison with CASE 2AH provides isothermal temperature coefficient.\\
    \hline
    3A & \gls{CR} inserted, otherwise same as CASE 1A. \\
    \hline
    4A & Disrete europia \gls{BP}, otherwise same as CASE 1A.\\
    \hline
    4AR & Discrete europia \gls{BP}, and \gls{CR} inserted, otherwise same as 
    CASE 1A. \\
    \hline
    5A & Integral (dispersed) europia \gls{BP}, otherwise same as CASE 1A. \\
    \hline
    6A & Increased \gls{HM} loading (4 to 8 layers of \gls{TRISO}), hence decreased C/HM 
    (from about 400 to about 200) and decreased specific power to 100 W/gU, 
    otherwise same as CASE 1A.\\
    \hline 
    7A & Fuel enrichment 19.75 wt\%, otherwise same as CASE 1A.\\
    \hline 
    \end{tabular}
\end{table}

\subsection{Phase I-B}
In phase I-B, the benchmark specifies analysis for 3 cases: 1B, 4B, and 7B. 
These are the same as cases 1A, 4A, and 7A, but with depletion steps added and
the critical spectrum assumption. 
The benchmark assumes that depletion occurs only in the fuel and \glspl{BP}. 
Table \ref{tab:phase1b} outlines the results required by the benchmark at specific 
depletion burnup steps. 

\begin{table}
    \centering
    \caption{Phase I-B required results at specific burnup steps \cite{noauthor_fluoride_nodate}.}
    \label{tab:phase1b}
    \includegraphics[width=0.8\linewidth]{phase1b.png}
  \end{table}

\section{Results}
Several organizations participated in the benchmark with various Monte Carlo
and Deterministic neutronics codes. 
We participated in the benchmark with the OpenMC Monte Carlo code 
\cite{romano_openmc_2013} and the ENDF/B-VII.1 material library 
\cite{chadwick_endf/b-vii.1_2011}.
The \texttt{fhr-benchmark} github repository contains all the results submitted 
by \gls{UIUC} for the \gls{FHR} benchmark. 
% needs citation both up and down
The benchmark used a phased blind approach, thus, participants were asked to 
submit phase I-A and I-B results without knowledge of other submissions. 
Petrovic et al describes the preliminary results of the benchmark results 
across the several institutions, and concludes that the overall observed agreement 
is satisfactory, although notable differences are identified in specific cases 
suggesting the need for further in-depth analysis of those cases. 
In the subsequent sections, we will share the results obtained by the \gls{UIUC}.  

\subsection{Phase I-A}
Table \ref{tab:phase1a-cases} reports phase I-A results for effective multiplication 
factor and reactivity coefficients. 
We ran the simulations on \gls{UIUC}'s BlueWaters supercomputer with 64 XE nodes, 
which have 32 cores each. 
To reduce statistical uncertainty of keff to $\sim$10pcm, each simulation took 
wall-clock-time of 2 to 5 hours. 
\begin{table}[H]
    \centering
    \onehalfspacing
    \caption{Phase I-A results cite github fhr bm.}
	\label{tab:phase1a-results}
    \footnotesize
    \begin{tabular}{cp{2.6cm}ccccc}
    \hline
    \textbf{Case} & \textbf{Summary} &\textbf{$k_{eff}$} & 
    \textbf{$\beta_{eff}$} & 
    \textbf{Fuel} $\frac{\Delta \rho}{\Delta T}$ & 
    \textbf{FliBe} $\frac{\Delta \rho}{\Delta T}$ & 
    \textbf{Graphite} $\frac{\Delta \rho}{\Delta T}$\\
    \hline 
    1A & Reference &1.39389$\pm$0.00010 & 0.006534$\pm$0.000001 & -2.24$\pm$0.15 & -0.15$\pm$0.15 & -0.68$\pm$0.15\\
    2AH & \gls{HZP} &1.40395$\pm$0.00010 & 0.006534$\pm$0.000001 & -3.14$\pm$0.15 & -0.20$\pm$0.14 & -0.85$\pm$0.14\\
    2AC & \gls{CZP} &1.41891$\pm$0.00010 & 0.006534$\pm$0.000001 & -3.36$\pm$0.14 & -0.11$\pm$0.14 & 0.07$\pm$0.14\\
    3A & \gls{CR} &1.03147$\pm$0.00011 & 0.006534$\pm$0.000001 & -4.03$\pm$0.28 & -0.83$\pm$0.27 & -3.18$\pm$0.29\\
    4A & Discrete \gls{BP} &1.09766$\pm$0.00010 & 0.006542$\pm$0.000001 & -4.06$\pm$0.24 & -1.55$\pm$0.23 & -6.51$\pm$0.24\\
    4AR & Discrete \gls{BP} +\gls{CR} &0.84158$\pm$0.00010 & 0.006553$\pm$0.000001 & -5.60$\pm$0.49 & -1.78$\pm$0.46 & -10.44$\pm$0.47\\
    5A & Dispersed \gls{BP} &0.79837$\pm$0.00009 & 0.006556$\pm$0.000001 & -5.09$\pm$0.40 & -4.87$\pm$0.40 & -22.99$\pm$0.38\\
    6A & Increased \gls{HM} &1.26294$\pm$0.00011 & 0.006556$\pm$0.000001 & -4.46$\pm$0.19 & 0.16$\pm$0.20 & -0.39$\pm$0.20\\
    7A & 19.75\% Enriched &1.50526$\pm$0.00010 & 0.006530$\pm$0.000001 & -2.49$\pm$0.13 & -0.12$\pm$0.12 & -0.62$\pm$0.12\\
    \hline
    \end{tabular}
\end{table}
Petrovic et al compared effective multiplication factor for all participants 
and cases, and reported that standard deviation between participants for each case 
was in the 231 to 514 pcm range, which is acceptable and notably close given a 
blind benchmark. 

Cases 2AH and 2AC are at zero power meaning that the fuel element is exactly 
critical but not producing any energy. 
For both cases, keff is higher than the reference case 1A which we attribute to 
lower fuel temperatures; at higher fuel temperatures, doppler broadening occurs 
which results in more neutron capture, thus, lowering keff. 
As expected, keff is lower for cases 3A, 4AR, and 5A compared to reference case 
1A since those cases introduced burnable poisons and control rods to the fuel 
element. 
Also as expected, keff is higher for case 7A compared to reference case 1A since 
the fuel has a higher enrichment. 
However, case 6A deviated from expectations with a lower keff despite an increase 
in \gls{HM} loading. 
This could be due reduced moderation and worsened fuel utilization brought 
about by self-shielding, demonstrating that an increase in fuel packing 
fraction does not always correspond with an increased keff. 

$\beta_{eff}$ increased for cases 4A, 4AR, 5A, and 6A.  