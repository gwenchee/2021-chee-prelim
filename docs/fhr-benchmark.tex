\chapter{Fluoride Salt Cooled High Temperature Reactor Benchmark}
\label{chap:fhr-benchmark}
% Main Gist 
% - The work I did for the FHR Benchmark
% Structure 
% - Specifications of benchmark problem
% - Results from benchmark
% Appendix? 
% - More details about the geometry 

The \gls{FHR} is a reactor concept that uses \gls{TRISO}-based fuel and a 
low-pressure liquid fluoride-salt coolant.
\gls{FHR} technology combines FLiBE coolant from \gls{MSR} technology and 
\gls{TRISO} particles from \gls{VHTR} technology to enable a reactor that 
has a low operating pressure, large thermal margin, and accident-tolerant 
qualities.
To address the technical challenges of \gls{AHTR} modeling, such as multiple 
heterogeneity and material cross-section data, the \gls{OECD}-\gls{NEA} and 
\gls{Georgia Tech} initiated the \gls{FHR} benchmark for the \gls{AHTR} design 
in 2019 \cite{noauthor_fluoride_nodate}. 
The \gls{AHTR} is a type of \gls{FHR} that has plate-based fuel in a hexagonal 
fuel assembly. 
In section \ref{sec:fhr}, we gave an overview of the \gls{FHR} concept, 
a high-level description of the \gls{AHTR} design, a review of previous efforts 
towards modeling these designs, and how these efforts led to the 
initiation of the benchmark. 

The \gls{FHR} benchmark has several phases, starting with single fuel assembly 
simulation without burnup, and gradually extending to full core depletion. 
Table \ref{tab:phases} outlines the complete and incomplete benchmark phases.

\begin{table}[H]
    \centering
    \onehalfspacing
    \caption{Phases of the \gls{FHR} benchmark \cite{noauthor_fluoride_nodate}.}
	\label{tab:phases}
    \footnotesize
    \begin{tabular}{lclc}
    \hline 
    \textbf{Phases}& \textbf{Sub-phases} & \textbf{Description} & \textbf{Completed?} \\
    \hline
    \multirow{ 3}{5cm}{\textbf{Phase I: Fuel assembly (2D/3D with depletion)}} & I-A & 2D model, steady state (no depletion) & \checkmark\\
    &I-B & 2D model depletion & \checkmark\\
    &I-C & 3D model depletion &\\
    \hline
    \multirow{2}{5cm}{\textbf{Phase II: 3D full core with depletion}}&II-A & Steady-state (no depletion) &\\
    &II-B & Depletion &\\
    \hline 
    \multirow{ 2}{5.5cm}{\textbf{Phase III: 3D full core with feedback \& multicycle analysis}}&III-A & Full core depletion with feedback &\\
    &III-B & Multicycle analysis &\\
    \hline
    \end{tabular}
\end{table}

In the subsequent sections, we will describe the benchmark's specifications for 
the \gls{AHTR} design and phase I. Then, we will share our phase I-A and I-B 
results. 

\section{Benchmark Specifications: AHTR Design}
The \gls{AHTR} has 3400 MWt thermal power and 1400 MW electric power 
\cite{varma_ahtr_2012}. 
Figure \ref{fig:reactor-schematic} shows the reactor schematic and a vertical 
cut of the reactor vessel. 
Figure \ref{fig:ahtr} shows the core configuration and fuel assembly. 
\begin{figure}[]
    \centering
    \includegraphics[width=0.9\linewidth]{reactor-schematic.png} 
    \caption{\acrlong{AHTR} schematic (left) and vessel (right) 
    \cite{noauthor_fluoride_nodate}.}
    \label{fig:reactor-schematic}
\end{figure}
The prismatic \gls{AHTR}'s fuel assembly is shown in Figure 
\ref{fig:ahtr-fuel-assembly}.  
\begin{figure}[]
    \centering
    \includegraphics[width=0.5\linewidth]{ahtr-fuel-element.png} 
    \caption{\acrlong{AHTR} fuel assembly with 18 fuel plates arranged in 
    three diamond-shaped sectors, with a central Y-shaped and external channel 
    graphite structure. Blue: FliBE coolant, Gray: graphite structural components, 
    Red: graphite fuel plank, Pink: graphite spacers, Green: graphite matrix 
    with embedded TRISO particles.}
    \label{fig:ahtr-fuel-assembly}
\end{figure}
It features plate-type fuel with hexagonal fuel assembly consisting of eighteen 
planks arranged in three diamond-shaped sectors, with a central Y-shaped 
structure and external channel (wrapper).
The diamond-shaped sections have 120-deg rotational symmetry with each other 
\cite{varma_ahtr_2012,ramey_monte_2018,noauthor_fluoride_nodate}. 
The fuel planks have semi-cylindrical spacers attached, their radius being 
equal to the coolant channel thickness. 

The external channel wrapper and structural Y-shape as seen in Figure 
\ref{fig:y-shape} are made of C-C composite and have extra notches for the 
fuel plates to slide in. 
\begin{figure}[]
    \centering
    \includegraphics[width=0.8\linewidth]{y-shape.png} 
    \caption{\acrlong{AHTR} fuel assembly's structural components 
    \cite{noauthor_fluoride_nodate}.}
    \label{fig:y-shape}
\end{figure}
The gap between the fuel assemblies and fuel plates are filled with \gls{FLiBe}
coolant. 
At the center of the Y-shape structure is the Y-shaped control rod slot, 
which contains \gls{FLiBe} coolant when the control blade is not in the slot
\cite{varma_ahtr_2012,ramey_monte_2018,noauthor_fluoride_nodate}.
Each fuel plank is made of an isostatically pressed carbon with fuel stripes 
on each outer side of the plank as seen in Figure \ref{fig:ahtr-fuel-plank}. 
\begin{figure}[]
    \centering
    \includegraphics[width=0.75\linewidth]{ahtr-fuel-plank.png} 
    \caption{\acrlong{AHTR}'s fuel plank, with magnification of 
    a spacer and segment of the fuel stripe with embedded TRISO particles.}
    \label{fig:ahtr-fuel-plank}
\end{figure}
The fuel stripes are prismatic regions composed of a graphite matrix filled with 
a cubic lattice of \gls{TRISO} particles with a 40\% packing fraction. 
The lattice is 210 \gls{TRISO} particles wide in x-direction, 4 particles deep in 
the y-direction, and 5936 particles tall in the z-direction. 
Each \gls{TRISO} particle has 5 layers: Oxycarbide fuel kernel, porous carbon 
buffer, inner pyrolytic carbon, silicon carbide layer, and the outer pyrolitic 
carbon as seen in Figure \ref{fig:ahtr-triso}
\begin{figure}[]
    \centering
    \includegraphics[width=0.35\linewidth]{ahtr-triso.png} 
    \caption{\acrlong{AHTR}'s TRISO particle schematic \cite{noauthor_fluoride_nodate}.}
    \label{fig:ahtr-triso}
\end{figure}

Burnable poisons and control rods for reactivity control are also included in 
some configurations of the \gls{AHTR}. 
The burnable poisons consist of europium oxide, $Eu_2O_3$, and have a discrete
or integral (dispersed) option. 
In the discrete option, small spherical $Eu_2O_3$ particles are stacked axially 
at 5 locations in each fuel plank as shown in Figure \ref{fig:discrete-poison}. 
\begin{figure}[]
    \centering
    \includegraphics[width=0.6\linewidth]{discrete-poison.png} 
    \caption{Placement of axial stacks of burnable poisons in the \acrlong{AHTR} 
    \cite{noauthor_fluoride_nodate}.}
    \label{fig:discrete-poison}
\end{figure}
In the integral option, $Eu_2O_3$ is homogenously mixed with the fuel plank 
graphite matrix (including the graphite in fuel stripes matrix, and the 
graphite in the plank ends indented to structural sides, but excluding the 
graphite in spacers and graphite in TRISO particles). 
Each control rod is uniformly composed of \gls{MHC} and is inserted into the 
Y-shaped control rod slot and displaces the \gls{FLiBe} that occupies the slot. 

\section{Benchmark Specifications: Phase I}
\label{sec:phase1}
Phase I of the \gls{FHR} benchmark consists of a steady-state 2D model 
(phase I-A) and depletion (phase I-B) of one \gls{FHR} fuel assembly. 
For a single fuel assembly, the internal 120-degree rotational symmetry is 
represented by periodic boundary conditions as seen in Figure \ref{fig:bc}. 
\begin{figure}[]
    \centering
    \includegraphics[width=0.5\linewidth]{bc.png} 
    \caption{Visualization of periodic boundary conditions for a single fuel 
    assembly \cite{noauthor_fluoride_nodate}.}
    \label{fig:bc}
\end{figure}
The benchmark required the following results for phases I-A and I-B:
\begin{enumerate}[label=(\alph*)]
    \item Effective multiplication factor 
    \item Reactivity coefficients ($\beta_{eff}$, fuel Doppler coefficient, FLiBe 
    temperature coefficient, graphite temperature coefficient)
    \item Tabulated fission source distribution, at several levels of granularity 
    (by fuel plate, by fuel stripe, by 1/5-th fuel stripe). 
    \item Neutron flux averaged over the whole model tabulated in 3 coarse energy groups. 
    \item Visualized distribution of the neutron flux distribution, in 3 coarse energy groups
    \item Neutron spectrum, fuel assembly average. Optional: by region.
\end{enumerate}

Next, we will report the equations we used to calculate the required results 
listed above. 

\subsubsection{Reactivity Coefficients (b)}
Effective delayed neutron fraction ($\beta_{eff}$) is the fraction of delayed 
neutrons in the core at creation at thermal energies. 
We assumed 1 energy group and 6 delayed groups for $\beta_{eff}$: 
\begin{align}
    \beta_{eff} = \sum_k \beta_k
\end{align}

Reactivity coefficient is the change in reactivity of the material per degree 
change in the material's temperature. 
We calculated each reactivity coefficient and its corresponding uncertainty 
with these equations: 
\begin{align}
    \frac{\Delta \rho}{\Delta T} &= 
    \frac{\rho_{T_{high}}-\rho_{T_{low}}}{T_{high}-T_{low}} [\frac{pcm}{K}] \\
    \delta \frac{\Delta \rho}{\Delta T} &= 
    \frac{\sqrt{\delta (\rho_{T_{high}})^2+(\delta \rho_{T_{low}})^2}}{T_{high}-T_{low}} [\frac{pcm}{K}] 
\end{align}

\subsubsection{Fission Source Distribution / Fission Density (c)}
We calculated \gls{FD} with OpenMC's \texttt{fission} score (f) for a region 
divided by the average \texttt{fission} score of all the regions:
\begin{align}
    FD_i &=  \frac{f_i}{f_{ave}} \\
    \intertext{where}
    f_i &= \mbox{Total fission reaction rate [reactions/src]} \nonumber \\
    f_{ave} &= \mbox{average of all $f_i$ [reactions/src]} \nonumber
\end{align}
The uncertainty calculations for $f_{ave}$ and $FD_i$: 
\begin{align}
    \delta f_{ave} &= \frac{1}{N}\sqrt{\sum_i^Nf_i^2} \\
    \delta FD_i &= |FD_i| \sqrt{(\frac{\delta f_i}{f_i})^2+(\frac{\delta f_{ave}}{f_{ave}})^2} \\
    \intertext{where}
    N &= \mbox{No. of fission score values} \nonumber
\end{align}

\subsubsection{Neutron Flux (d, e, f)}
OpenMC's \texttt{flux} score are in [$\frac{n * cm}{src}$] units. 
For the benchmark, we converted flux to [$\frac{n}{cm^2s}$] units
using the following equations:  

\begin{align}
    \Phi_c &= \frac{N* \Phi_o}{V} \\
    N &= \frac{P*\nu}{Q*k} \\
    \intertext{where}
    \Phi_c &= \mbox{Converted Flux [$\frac{neutrons}{cm^2s}$]} \nonumber \\ 
    \Phi_o &= \mbox{Original Flux [$\frac{neutrons* cm}{src}$]} \nonumber \\
    N &= \mbox{Normalization factor [$\frac{src}{s}$]} \nonumber \\
    V &= \mbox{Volume of fuel assembly [$cm^3$]} \nonumber \\
    P &= \mbox{Power [$\frac{J}{s}$]} \nonumber \\
    \nu &= \mbox{$\frac{\nu_f}{f}$ [$\frac{neutrons}{fission}$]} \nonumber \\
    Q &= \mbox{Energy produced per fission [$\frac{J}{fission}$]} = \mbox{$3.2044*10^{-11}$ J per $U_{235}$ fission} \nonumber \\
    k &= \mbox{$k_{eff}$ [$\frac{neutrons}{src}$]} \nonumber 
\end{align}
Flux standard deviation: 
\begin{align}
    \delta \Phi_c = \Phi_c * 
    \sqrt{(\frac{\delta \Phi_o}{\Phi_o})^2+ (\frac{\delta \nu_f}{\nu_f})^2 
    + (\frac{\delta k}{k})^2 + (\frac{\delta f}{f})^2}
\end{align}
We calculated reactor power based on the given reference specific power 
($P_{sp}$) of 200 $\frac{W}{gU}$. 
\begin{align}
    P &= P_{sp} * V_F * \rho_F * \frac{wt\%_{U}}{100} \\
    \intertext{where}
    V_F &= \mbox{volume of fuel [$cm^3$]} = \frac{4}{3} \pi r_f^3 * N_{total} \nonumber \\
    r_f &= \mbox{radius of fuel kernel} \nonumber\\
    N_{total} &= \mbox{total no. of TRISO particles in fuel assembly} \nonumber\\ 
    &= 101 * 210 * 4 * 2 * 6 * 3 \nonumber\\
    \rho_F &= \mbox{density of fuel [$g/cc$]} \nonumber \\
    wt\%_{U} &= \frac{at\%_{U235} * AM_{U235} + at\%_{U238} * AM_{U238}}{\sum (at\%_i * AM_i)} * 100 \nonumber\\
    AM &= \mbox{atomic mass} \nonumber
\end{align}

\subsection{Benchmark Specifications: Phase I-A}
For phase I-A, the benchmark specifies that each participant must produce a 
steady-state 2D model of one fresh fuel assembly for 9 cases, 
and report the required results listed in Section \ref{sec:phase1}.  
Table \ref{tab:phase1a-cases} describes each case. 
\begin{table}[H]
    \centering
    \onehalfspacing
    \caption{Description of cases in Phase I-A of the \gls{FHR} benchmark \cite{noauthor_fluoride_nodate}.}
	\label{tab:phase1a-cases}
    \footnotesize
    \begin{tabular}{p{0.05\textwidth}|p{0.9\textwidth}}
    \hline 
    \textbf{Case} & \textbf{Description} \\
    \hline
    1A & Reference case. Hot full power (HFP), with temperatures of 1110K for 
    fuel kernel and 948K for coolant and all other materials (including TRISO 
    particle layers other than fuel kernel). Nominal (cold) dimensions, 
    9 wt\% enrichment, no \gls{BP}, \glspl{CR} out.\\
    \hline
    2AH & \gls{HZP} with uniform temperature of 948 K, 
    otherwise same as CASE 1A. Comparison with CASE 1A provides HZP-to-HFP power 
    defect.\\
    \hline 
    2AC & \gls{CZP}. Same as CASE 2AH, but with uniform temperature 
    of 773 K. Comparison with CASE 2AH provides isothermal temperature coefficient.\\
    \hline
    3A & \gls{CR} inserted, otherwise same as CASE 1A. \\
    \hline
    4A & Disrete europia \gls{BP}, otherwise same as CASE 1A.\\
    \hline
    4AR & Discrete europia \gls{BP}, and \gls{CR} inserted, otherwise same as 
    CASE 1A. \\
    \hline
    5A & Integral (dispersed) europia \gls{BP}, otherwise same as CASE 1A. \\
    \hline
    6A & Increased \gls{HM} loading (4 to 8 layers of \gls{TRISO}), hence decreased C/HM 
    (from about 400 to about 200) and decreased specific power to 100 W/gU, 
    otherwise same as CASE 1A.\\
    \hline 
    7A & Fuel enrichment 19.75 wt\%, otherwise same as CASE 1A.\\
    \hline 
    \end{tabular}
\end{table}

\subsection{Benchmark Specifications: Phase I-B}
For phase I-B, the benchmark specifies that each participant must produce 
depletion results for 3 cases: 1B, 4B, and 7B. 
These are the same as cases 1A, 4A, and 7A, but with depletion steps added and
the critical spectrum assumption. 
The benchmark assumes that depletion occurs only in the fuel and \glspl{BP}. 
Table \ref{tab:phase1b} outlines the results required by the benchmark at specific 
depletion burnup steps for all three cases. 

\begin{table}
    \centering
    \caption{\gls{FHR} benchmark Phase I-B required results at specific burnup 
    steps \cite{noauthor_fluoride_nodate}.}
    \label{tab:phase1b}
    \includegraphics[width=0.8\linewidth]{phase1b.png}
  \end{table}

\section{Results}
Several organizations participated in the benchmark with various Monte Carlo
and Deterministic neutronics codes, such as Serpent \cite{leppanen_serpent_2014}, 
OpenMC \cite{romano_openmc_2013}, and WIMS \cite{lindley_current_2017}. 
We participated in the benchmark with the OpenMC Monte Carlo code 
\cite{romano_openmc_2013} and the ENDF/B-VII.1 material library 
\cite{chadwick_endf/b-vii.1_2011}.
The \texttt{fhr-benchmark} github repository contains all the results submitted 
by \gls{UIUC} for the \gls{FHR} benchmark. 
% needs citation both up and down
The benchmark used a phased blind approach, thus, participants were asked to 
submit phase I-A and I-B results without knowledge of other submissions. 
Petrovic et al describes the preliminary results of the benchmark results 
across the several institutions, and concludes that the overall observed agreement 
is satisfactory, although notable differences are identified in specific cases 
suggesting the need for further in-depth analysis of those cases. 
In the subsequent sections, we will share the results obtained by \gls{UIUC}.  

\subsection{Results: Phase I-A}
Petrovic et al compared effective multiplication factor for all participants 
and phase I-A cases in the \gls{FHR} benchmark, and reported that standard deviation 
between participants for each case was in the 231 to 514 pcm range, which is 
acceptable and notably close given a blind benchmark. % CITE!! 
This assures us that our results for phase I-A are acceptable and in agreement 
with other participants in the benchmark. 
Next, we will present our results for phase I-A and describe and explain the 
observed trends.

Table \ref{tab:phase1a-cases} reports phase I-A results for effective multiplication 
factor and reactivity coefficients. 
We ran the simulations on \gls{UIUC}'s BlueWaters supercomputer \cite{ncsa_about_2017}
with 64 XE nodes, which have 32 cores each. 
To reduce statistical uncertainty of keff to $\sim$10pcm, we ran each simulation 
with 500 active cycles, 100 inactive cycles, and 200000 neutrons. 
Each simulation took \gls{WCT} ranging from 2 to 5 hours. 
\begin{table}[H]
    \centering
    \onehalfspacing
    \caption{\gls{FHR} Benchmark Phase I-A results cite github fhr bm.}
	\label{tab:phase1a-results}
    \footnotesize
    \begin{tabular}{cp{2.7cm}cccccc}
    \hline
    \textbf{Case} & \textbf{Summary} & \textbf{WCT [hr]} & \textbf{$k_{eff}$} & 
    \textbf{$\beta_{eff}$}* & 
    \textbf{Fuel} $\frac{\Delta \rho}{\Delta T}$ & 
    \textbf{FliBe} $\frac{\Delta \rho}{\Delta T}$ & 
    \textbf{Graphite} $\frac{\Delta \rho}{\Delta T}$\\
    \hline 
    1A & Reference &2.82&1.39389$\pm$0.00010 & 0.006534 & -2.24$\pm$0.15 & -0.15$\pm$0.15 & -0.68$\pm$0.15\\
    2AH & \gls{HZP} &2.82&1.40395$\pm$0.00010 & 0.006534 & -3.14$\pm$0.15 & -0.20$\pm$0.14 & -0.85$\pm$0.14\\
    2AC & \gls{CZP} &2.75&1.41891$\pm$0.00010 & 0.006534 & -3.36$\pm$0.14 & -0.11$\pm$0.14 & 0.07$\pm$0.14\\
    3A & \gls{CR} &2.49&1.03147$\pm$0.00011 & 0.006534 & -4.03$\pm$0.28 & -0.83$\pm$0.27 & -3.18$\pm$0.29\\
    4A & Discrete \gls{BP} &5.08&1.09766$\pm$0.00010 & 0.006542 & -4.06$\pm$0.24 & -1.55$\pm$0.23 & -6.51$\pm$0.24\\
    4AR & Discrete \gls{BP} + \gls{CR} &4.59&0.84158$\pm$0.00010 & 0.006553 & -5.60$\pm$0.49 & -1.78$\pm$0.46 & -10.44$\pm$0.47\\
    5A & Dispersed \gls{BP} &2.33&0.79837$\pm$0.00009 & 0.006556 & -5.09$\pm$0.40 & -4.87$\pm$0.40 & -22.99$\pm$0.38\\
    6A & Increased \gls{HM} &3.52&1.26294$\pm$0.00011 & 0.006556 & -4.46$\pm$0.19 & 0.16$\pm$0.20 & -0.39$\pm$0.20\\
    7A & 19.75\% Enriched &2.21&1.50526$\pm$0.00010 & 0.006530 & -2.49$\pm$0.13 & -0.12$\pm$0.12 & -0.62$\pm$0.12\\
    \hline
    \multicolumn{5}{l}{* All $\beta_{eff}$ values have an uncertainty of 0.000001.} 
    \end{tabular}
\end{table}

Cases 2AH and 2AC are at zero power meaning that the fuel assembly is exactly 
critical but not producing any energy. 
For both cases, keff is higher than the reference case 1A which we attribute to 
lower fuel temperatures; at higher fuel temperatures, doppler broadening occurs 
which results in more neutron capture, thus, lowering keff. 
As expected, keff is lower for cases 3A, 4AR, and 5A compared to reference case 
1A since those cases introduced burnable poisons and control rods to the fuel 
assembly. 
Also as expected, keff is higher for case 7A compared to reference case 1A since 
the fuel has a higher enrichment. 
However, case 6A deviated from expectations with a lower keff despite an increase 
in \gls{HM} loading. 
This could be due reduced moderation and worsened fuel utilization brought 
about by self-shielding, demonstrating that an increase in fuel packing 
fraction does not always correspond with an increased keff. 

$\beta_{eff}$ increased by 10-20pcm for cases 4A, 4AR, 5A, and 6A, compared to
reference case 1A.
This is attributed to the introduction of control rods and poisons, which 
shifts the average neutron velocity to higher values resulting in decreased
thermal fission and increased fast fission\cite{torabi_neutronic_2018}.
Table \ref{tab:phase1a-results} reports that most of the temperature coefficients 
are negative, which exemplifies the \gls{AHTR}'s passive safety behavior. 
Negative reactivity feedback results in a self-regulating reactor; if the reactor's 
power rises, resulting in temperature increase, the negative reactivity will in turn 
reduce power. 

Figure \ref{fig:phase1a-c} shows the fission source distribution by 
$1/5^{th}$ fuel stripe for cases 1A and 3A. 
\begin{figure}[]
    \centering
    \includegraphics[width=0.49\linewidth]{p1a_c1a_c.png} 
    \includegraphics[width=0.49\linewidth]{p1a_c3a_c.png} 
    \caption{Fission Source Distribution per $1/5^{th}$ fuel stripe for \gls{FHR} 
    Benchmark's Phase I-A Case 1A (left) and Case 3A (right).}
    \label{fig:phase1a-c}
\end{figure}
Case 4AR has a similar fission source distribution as case 3A, since both 
cases have control rod insertion. 
All other cases have similar fission source distributions to case 1A. 
For case 1A, intuitively, we would assume that highest fission source in stripes 
corresponding to the centre of the diamond fuel segment, however, the opposite is 
true. 
Power peaking occurs on exterior stripes and is minimum on the interior stripes.  
Gentry et al \cite{gentry_development_2016} reported a similar phenomena of 
power peaking towards the exterior of the lattice cell closest to the Y-shaped 
carbon support structure where the thermal flux is most elevated, and the 
complement to the peaking is found in the interiors of the lattice tri-sections. 
This fission source distribution is caused by  diminished resonance escape 
probability in the interior due to higher relative fuel-to-carbon volume ratio. 
This phenomenon is more exaggerated in cases 6A and 7A since there is a higher 
fuel-to-carbon ratio. 
For case 3A, the fission source is lower in the $1/5^{th}$ stripes closer to 
the control rod.  

Figure \ref{fig:phase1a-d} shows average neutron flux in the fuel assembly in 
three coarse energy groups. 
\begin{figure}[]
    \centering
    \includegraphics[width=\linewidth]{phase1a-d-flux.png} 
    \caption{Neutron Flux, averaged over the whole model, tabulated in 3 coarse 
    energy groups for each Phase I-A case. }
    \label{fig:phase1a-d}
\end{figure}
Most of the cases have the most flux in the intermediate group, followed by 
the thermal group, and least flux in the fast group.    
Figure \ref{fig:phase1a-e} shows the neutron flux distribution for case 1A, 
3A, and 6A for 3 coarse energy groups. 
\begin{figure}[]
    \centering
    \includegraphics[width=0.9\linewidth]{phase1a-e-c1a.png} 
    \includegraphics[width=0.9\linewidth]{phase1a-e-c3a.png} 
    \includegraphics[width=0.9\linewidth]{phase1a-e-c6a.png} 
    \caption{Neutron Flux distribution in 100 by 100 mesh for 3 coarse 
    energy groups: case 1A (above), case 3A (middle), case 6A (below) }
    \label{fig:phase1a-e}
\end{figure}
For all three cases, fast flux peaks in the diamond-shaped sectors that contain the 
fuel stripes whereas thermal flux peaks outside of the diamond-shaped sectors. 
This is attributed to fission occurring at thermal energies in the fuel stripe 
area. 
For case 3A, thermal and intermediate neutron flux is depressed in the control 
rod region of the fuel assembly.  
Case 6A has increased \gls{HM}, thus, we see that fast flux peaking and the thermal 
flux dip in the fuel stripe area is more pronounced than case 1A. 
Figure \ref{fig:phase1a-f} shows the neutron spectrum for cases 1A and 6A. 
\begin{figure}[]
    \centering
    \includegraphics[width=0.49\linewidth]{p1a_c1a_f.png} 
    \includegraphics[width=0.49\linewidth]{p1a_c6a_f.png} 
    \caption{Neutron Spectrum for \gls{FHR} Benchmark's Phase I-A Case 1A 
    (left) and Case 6A (right).}
    \label{fig:phase1a-f}
\end{figure}
Case 7A has a similar neutron spectrum as case 6A, since both cases have 
control rod insertion. 
All other cases have similar neutron spectrum to case 1A.
The neutron spectrum is faster for case 6a and case 7a due to more heavy metal 
loading and higher enrichment.  

\subsection{Results: Phase I-B}


\section{Summary}
% more enrichment / more HM does not mean higher keff, there is shielding 
% effects, thus, this leads us to believe that the phenomena is not as expected. 

This chapter described the \gls{FHR} benchmark specifications and the results 
obtained so far by the UIUC team. 
Unanticipated results such as a lower keff for the \gls{AHTR} configuration with 
higher \gls{HM} loading gave insight to how increased fuel packing does not always 
correspond with increased keff due to self-shielding.
This hints to the possibility of minimizing fuel required by optimizing for 
inhomogeneous fuel distributions within the core. 
This will be further explored in the next chapters. 

