\chapter{Introduction}
\label{chap:intro}
% Main Gist 
% - Reactor design optimization is a highly iterative process which requires
%   back and forth between neutronics and thermal hydraulics designers. 
%   It requires years of experience to have intuition on a global optimum 
%   solution. AI applications can help with this. 
% - I will be applying AI methods to the salt-cooled reactor type because its 
%   an awesome reactor type. 
% Structure 
% - Motivation 
% - Outline of dissertation
%   - FHR Benchmark 
%   - Create a python package to couple genetic algorithm to neutronics / 
%     hydraulics code 
%   - Demonstrate realm with openmc 
%   - Demonstrate realm with moltres
%   - Evaluate the moltres' heat transfer interface and improve on it


\section{Motivation}
The impact of climate change on natural and human systems brought about by 
elevated \gls{GHG} concentrations is increasingly apparent such as increases 
in global average surface temperatures, sea levels, and larger climate extremes 
\cite{noauthor_climate_2018}.
Energy use and production contribute to two-thirds of the total \gls{GHG}
emissions \cite{noauthor_climate_2018}.
Furthermore, as the human population increases and previously under-developed 
nations rapidly urbanize, global energy demand is forecasted to increase.  
Energy generation technology selection profoundly impacts climate change via 
growing energy demand. 
Large scale deployment of emissions-free nuclear power plants could 
significantly reduce GHG production \cite{noauthor_climate_2018}.  
However, large scale nuclear power deployment faces challenges of cost and 
safety \cite{petti_future_2018}. 
The nuclear power industry must overcome these challenges to ensure continued 
global use and expansion of nuclear energy technology to provide low-carbon 
electricity worldwide to battle climate change.
To enhance the role of nuclear energy in our global energy 
eco-system, the Generation IV International Forum was created to lead and plan 
the research and development roadmap to support a new generation IV of innovative 
nuclear energy systems \cite{gif_technology_2002}.
Generation IV nuclear systems' goals are defined in four areas: sustainability, 
economics, safety and reliability, proliferation resistance and physical 
protection \cite{gif_technology_2002}. 
Table \ref{tab:goals-gen4} summarizes the goals in each area. 

\begin{table}[]
    \centering
    \onehalfspacing
    \caption{Goals of Generation IV Nuclear Systems \cite{gif_technology_2002,
    behar_technology_2014}}
	\label{tab:goals-gen4}
    \small
    \begin{tabular}{l|l}
    \hline
                               \textbf{Area} & \textbf{Goals} \\ \hline
    Sustainability   & - Have a positive impact on the environment through the displacement of \\
    & polluting energy and transportation sources by nuclear electricity generation \\
    & and nuclear-produced hydrogen \\ 
    & - Promote long-term availability of nuclear fuel \\
    & - Minimize volume, lifetime, and toxicity of nuclear waste \\ \hline
    Economics & - Have a life cycle and energy production cost advantage over other energy \\
    & sources \\ 
    & - Reduce economic risk to nuclear projects by developing plants using \\
    & innovative fabrication and construction techniques \\ \hline
    Safety and Reliability   & - Increase the use of robust designs, inherent and transparent safety features\\
    & that can be understood by non-experts \\ 
    & - Enhance public confidence in the safety of nuclear energy \\\hline
    Proliferation Resistance & - Provide continued effective proliferation resistance of nuclear energy \\
    and Physical Protection & systems through improved design features and other measures \\ 
    & - Increase the robustness of new facilities \\ \hline
    \end{tabular}
    \end{table}

An evaluation and selection methodology was developed based on the goals, 
culminating in selecting six Generation IV systems: \gls{GFR}, 
\gls{LFR}, \gls{MSR}, \gls{SFR}, \gls{SCWR}, and \gls{VHTR} \cite{gif_technology_2002}. 
The reactor systems of interest in this proposed work are the \gls{MSR} and 
\gls{VHTR} systems. 
The MSR system produces fission power in a circulating molten salt fuel mixture. 
It has a closed fuel cycle tailored to the efficient utilization of plutonium 
and minor actinides. 
Molten fluoride salts have very low vapor pressure, which reduces stress on the 
system. 
MSR systems also have inherent system safety due to fail-safe drainage, 
passive cooling, and a low inventory of volatile fission products in the fuel. 
The MSR system is top-ranked in sustainability because of its closed fuel cycle 
and excellent waste burndown performance. 
It rates good in safety and proliferation resistance and physical protection 
due to its inherent safety features and rates neutral in economics because of
its large number of subsystems \cite{gif_technology_2002}.  
The \gls{VHTR} system has a once-through uranium cycle and is primarily aimed at 
high-temperature heat applications, such as hydrogen production. 
It is a graphite-moderated, helium-cooled reactor that uses \gls{TRISO} fuel, 
which does not degrade at high burnup and temperature.  
The \gls{VHTR} system is highly ranked in economics, because of its high hydrogen 
production efficiency, safety and reliability, because of the fuel and reactor's 
inherent safety features. 
It is rated good in proliferation resistance and physical protection and 
neutral in sustainability because of its open fuel cycle \cite{gif_technology_2002}. 

% maybe need a better connection from MSR VHTR to FHR? 
In the proposed work, we will be exploring the \gls{FHR} reactor concept, which 
is a combination of the best aspects of \gls{MSR} and \gls{VHTR} technologies. 
The \gls{FHR} uses high-temperature coated-particle fuel (similar to the \gls{VHTR}) 
and a low-pressure liquid fluoride-salt coolant (similar to the \gls{MSR})
\cite{forsberg_fluoride-salt-cooled_2012,facilitators_fluoride-salt-cooled_2013}.

In recent years, additive manufacturing technology has advanced and altered the
manufacturing and design of components \cite{simpson_considerations_2019}. 
The automotive and aircraft industry have successfully fabricated parts, with 
key additive manufacturing technologies relevant to nuclear reactor core 
structures \cite{murr_frontiers_2016}.  
Successful examples of additive manufacturing applied in the aircraft industry 
are Boeing’s use of additive manufacturing to reduce weight in the 787 Dreamliner
\cite{noauthor_printed_2017} and SES-15 spacecraft \cite{noauthor_boeing_nodate}. 
Like the nuclear industry, the aerospace industry is highly regulated; thus, 
successful additive manufacturing applications in the aerospace industry are 
promising for the nuclear industry. 
Using additive manufacturing to fabricate nuclear reactor components will 
drastically reduce cost, timelines, increase safety, and performance by 
tailoring local material properties and redesigning geometries for optimal load paths 
\cite{simpson_considerations_2019}. 

With further advancement of these additive manufacturing technologies, a reactor 
core could be 3D printed in the near future. 
\gls{ORNL} is leading this initiative through the 2019 \gls{TCR} Demonstration 
Program. 
The \gls{TCR} program will leverage recent scientific achievements in advanced 
manufacturing, nuclear materials, machine learning, computational modeling 
and simulation to build a microreactor. 
The program targets to design, manufacture, and operate a demonstration reactor 
by 2023 \cite{terrani_transformational_2019}. 
% explore funky designs :D 
Applying additive manufacturing to nuclear reactor design will enable complex 
reactor geometries that are not limited by previous manufacturing constraints,
opening the door for a re-examination of nuclear reactor optimization 
\cite{sobes_artificial_2020}. 
Optimization efforts towards classically-manufactured nuclear reactors and now
3D printed nuclear reactors have focused on uniform shapes such as radius of the 
core, height of cylinder, enrichment of fuel, etc 
\cite{sobes_artificial_2020,sacco_two_2006,kumar_new_2015,pereira_parallel_2008}. 
Leveraging additive manufacturing technology enables us to surpass classical 
manufacturing constraints such as straight fuel channels or homogenous fuel 
enrichment, and look forward to optimizing with non-uniform channel shapes, 
inhomogeneous fuel enrichment throughout the core, etc.

Multi-objective design problems inevitably require a trade-off between 
desirable attributes \cite{byrne_evolving_2014,simon_sciences_2019}. 
There are many trade-offs in nuclear reactor design; one example is the 
trade-off between neutron economy and fuel enrichment.
A reactor design must have sufficient neutron economy to ensure criticality but 
also have a low fuel enrichment to reduce proliferation risk. 
Conflicting objectives means that there is no one perfect solution but a set
of equally optimal solutions \cite{byrne_evolving_2014}.
Multi-objective problems are challenging to optimize; therefore, they cannot be 
handled by classical optimization methods such as gradient methods because only 
the local optimum will be found \cite{renner_genetic_2003}. 
Evolutionary algorithms have proven successful methods to optimize 
multi-objective problems \cite{krish_practical_2011} as 
they can find a solution near the global optimum \cite{renner_genetic_2003}.
They also take advantage of parallel systems for reduced computational cost.
The most popular evolutionary algorithms used to solve multi-objective 
problems are genetic algorithms 
\cite{byrne_evolving_2014, krish_practical_2011}. 
Genetic algorithms imitate natural selection to evolve solutions 
by (1) maintaining a population of solutions, (2) allowing 
fitter solutions to reproduce, and (3) letting lesser fit solutions die off, 
resulting in final solutions that are better than the previous generations 
\cite{renner_genetic_2003}. 
We will pursue the evolutionary algorithm optimization technique 
in this work. 

Therefore, in this work, we propose designing an optimization tool that uses 
the evolutionary algorithm optimization technique with nuclear transport and 
thermal-hydraulics software. 
This tool will be used to explore non-uniform FHR reactor core parameters, now 
possible with additive manufacturing technology, to optimize reactor systems 
fully. 

\section{Objectives}
The proposed work's main objectives were developed based on leveraging 
open-source artificial intelligence tools with validated open-source nuclear 
transport and thermal-hydraulics software to create an open-source tool to 
generate optimal reactor designs quickly. 
Accordingly, the objectives are listed below. 

\vspace{0.2cm} 
\noindent
\textbf{Model Fluoride-Salt-Cooled High-Temperature Reactor with established 
nuclear transport and thermal-hydraulics software}.
To demonstrate success in modeling the \gls{FHR} with nuclear transport and 
thermal-hydraulics software before using the optimization tool, we will 
participate in the \gls{OECD} \gls{NEA}'s \gls{FHR} benchmark 
\cite{noauthor_fluoride_nodate}. 

\vspace{0.2cm} 
\noindent
\textbf{Develop a tool that applies evolutionary algorithms to nuclear 
reactor design optimization}. 
This tool will not re-invent the wheel; it will utilize a well-documented 
and validated open-source evolutionary algorithm python package with established 
nuclear transport and thermal-hydraulics software. This tool will run parallel on 
\gls{HPC} machines. This tool will be open-source and follow the rules for ensuring 
reproducibility, effectiveness, and usability 
\cite{list_ten_2017,osborne_ten_2014,sandve_ten_2013}. 

\vspace{0.2cm} 
\noindent
\textbf{Demonstrate nuclear reactor design optimization with the optimization tool 
for a neutronics problem}. 
We will demonstrate successful implementation of the optimization tool with the 
nuclear transport software by optimizing a simple \gls{FHR} model for a single 
objective function. 

\vspace{0.2cm} 
\noindent
\textbf{Demonstrate hyperparameter tuning with the optimization tool for neutronics problem}.
Hyperparameter selection will impact the effectiveness of the algorithm 
for our problem. 
Therefore, we must conduct hyperparameter tuning to find hyperparameters that work 
best for our problem. 

\vspace{0.2cm} 
\noindent
\textbf{Demonstrate nuclear reactor design optimization and hyperparameter 
search with the optimization tool for a neutronics and thermal-hydraulics problem}.
We will demonstrate successful implementation of the optimization tool and 
hyperparameter tuning with the nuclear transport and thermal-hydraulics tools 
for a \gls{FHR} model.  


\section{Outline}
This document outlines the motivation, preliminary work, and future work proposed 
towards developing an open-source optimization tool that applies evolutionary 
algorithms to nuclear transport and thermal-hydraulics software to optimize 
nuclear reactor design. 
Chapter 1 describes the motivation and objectives of the proposed work. 
Chapter 2 will present a literature review that organizes and reports on previous 
relevant work. 
Chapter 3 will describe the \gls{FHR} benchmark specifications and the results 
obtained thus far. 
Chapter 4 will detail the computational design of the REALM python package. 
Chapter 5 will demonstrate nuclear reactor optimization with REALM. 
Chapter 6 will summarize the remaining future work. 
