\chapter{Introduction}
% Main Gist 
% - Reactor design optimization is a highly iterative process which requires
%   back and forth between neutronics and thermal hydraulics designers. 
%   It requires years of experience to have intuition on a global optimum 
%   solution. AI applications can help with this. 
% - I will be applying AI methods to the salt-cooled reactor type because its 
%   an awesome reactor type. 
% Structure 
% - Motivation 
% - Outline of dissertation
%   - FHR Benchmark 
%   - Create a python package to couple genetic algorithm to neutronics / 
%     hydraulics code 
%   - Demonstrate realm with openmc 
%   - Demonstrate realm with moltres
%   - Evaluate the moltres' heat transfer interface and improve on it


\section{Motivation}
The impact of climate change on natural and human systems is increasingly 
apparent \cite{noauthor_climate_2018}.
Increases in global average surface temperatures, sea levels, and larger climate 
extremes are a few consequences brought on by elevated \gls{GHG} concentrations 
\cite{noauthor_climate_2018}.
Energy use and production contribute to two-thirds of the total \gls{GHG}
emissions \cite{noauthor_climate_2018}.
Furthermore, as the human population increases and previously under-developed 
nations rapidly urbanize, global energy demand is forecasted to increase.  
Energy generation technology selection profoundly impacts climate change via 
growing energy demand. 
Large scale deployment of emissions free nuclear power plants could 
significantly reduce GHG production \cite{noauthor_climate_2018}.  
However, large scale nuclear power deployment faces challenges of cost and 
safety \cite{petti_future_2018}. 
The nuclear power industry must overcome these challenges to ensure continued 
global use and expansion of nuclear energy technology to provide low-carbon 
electricity in the global effort to battle climate change. 

In an effort to enhance the role of nuclear energy in our global energy 
eco-system, the Generation IV International Forum was created to lead and plan 
the research and development roadmap to support a new generation IV of innovative 
nuclear energy systems \cite{gif_technology_2002}.
The goals of Generation IV nuclear systems are defined in four areas: 
sustainability, economics, safety and reliability, and proliferation resistance 
and physical protection \cite{gif_technology_2002}. 
Table \ref{tab:goals-gen4} summarizes the goals in each area. 

\begin{table}[]
    \centering
    \onehalfspacing
    \caption{Goals of Generation IV Nuclear Systems \cite{gif_technology_2002,
    behar_technology_2014}}
	\label{tab:goals-gen4}
    \small
    \begin{tabular}{l|l}
    \hline
                               \textbf{Area} & \textbf{Goals} \\ \hline
    Sustainability   & - Have a positive impact on the environment through displacement of \\
    & polluting energy and transportation sources by nuclear electricity generation \\
    & and nuclear-produced hydrogen \\ 
    & - Promote long-term availability of nuclear fuel \\
    & - Minimize volume, lifetime and toxicity of nuclear waste \\ \hline
    Economics & - Have a life cycle and energy production cost advantage over other energy \\
    & sources \\ 
    & - Reduce economic risk to nuclear projects by developing plants using \\
    & innovative fabrication and construction techniques \\ \hline
    Safety and Reliability   & - Increase the use of inherent safety features, robust designs, and \\
    & transparent safety features that can be understood by non experts \\ 
    & - Enhance public confidence in the safety of nuclear energy \\\hline
    Proliferation Resistance & - Provide continued effective proliferation resistance of nuclear energy \\
    and Physical Protection & systems through improved design features and other measures \\ 
    & - Increase robustness of new facilities \\ \hline
    \end{tabular}
    \end{table}

Based on the goals, an evaluation and selection methodology was developed which 
culminated in the selection of six Generation IV systems: \gls{GFR}, 
\gls{LFR}, \gls{MSR}, \gls{SFR}, \gls{SCWR}, and \gls{VHTR} \cite{gif_technology_2002}. 
The reactor systems of interest in this dissertation are the \gls{MSR} and \gls{VHTR} systems. 
The \gls{GFR} system features a fast-neutron spectrum and an outlet temperature of 
$\sim 850 ^{\circ}C$. 
Several fuel forms are considered in this system for their potential to operate 
at high temperatures and ensure an excellent retention of fission products:
composite ceramic fuel, advanced fuel particles, etc \cite{gif_technology_2002}.   
The \gls{MSR} system ... 
The \gls{VHTR} system ... 
% Describe MSR and VHTR systems 

In this dissertation, I will be exploring the \gls{FHR} reactor concept. 
The \gls{FHR} uses high-temperature coated-particle fuel and a low pressure 
liquid fluoride-salt coolant, which is a combination of the best aspects of 
\gls{MSR} and \gls{HTGR} (also known as \gls{VHTR}) technologies
\cite{forsberg_fluoride-salt-cooled_2012,facilitators_fluoride-salt-cooled_2013}.

In recent years, additive manufacturing technology has advanced and  
altered the way in which components are designed and manufactured 
\cite{simpson_considerations_2019}. 
Key additive manufacturing technologies relevant to nuclear reactor core 
structures are \gls{SLM}, \gls{EBM}, and \gls{L-DED}. 
The \gls{SLM} and \gls{EBM} techniques have been used for automotive and aircraft component 
fabrication \cite{murr_frontiers_2016}.  
Successful examples of additive manufacturing applied in the the aircraft industry 
are Boeing’s use of additive manufacturing to reduce weight in the 787 Dreamliner
\cite{noauthor_printed_2017} and SES-15 spacecraft \cite{noauthor_boeing_nodate}.  
Using additive manufacturing to fabricate nuclear reactor components will 
drastically reduce cost, timelines, increase safety and performance by 
tailoring local material properties and redesigning geometries for optimal load paths 
\cite{simpson_considerations_2019}. 
With further advancement of these additive manufacturing technologies, a reactor 
core could be 3D printed in the near future. 
\gls{ORNL} is leading this initiative through the 2019 \gls{TCR} Demonstration 
Program. 
The program targets to design, manufacture, and operate a demonstration reactor 
by 2023 \cite{terrani_transformational_2019}. 
% explore funky designs :D 

\section{Objectives}
This dissertation's objectives were developed based on leveraging open-source 
artificial intelligence tools with validated open-source nuclear transport and 
thermal hydraulics software to create an open-source tool to easily generate 
optimal reactor designs. 
Accordingly, an outline of the steps to accomplish the motivation and goals 
of this dissertation are listed below. 
