\chapter{Introduction}
\label{chap:intro}
% Main Gist 
% - Reactor design optimization is a highly iterative process which requires
%   back and forth between neutronics and thermal hydraulics designers. 
%   It requires years of experience to have intuition on a global optimum 
%   solution. AI applications can help with this. 
% - I will be applying AI methods to the salt-cooled reactor type because its 
%   an awesome reactor type. 
% Structure 
% - Motivation 
% - Outline of dissertation
%   - FHR Benchmark 
%   - Create a python package to couple genetic algorithm to neutronics / 
%     hydraulics code 
%   - Demonstrate realm with openmc 
%   - Demonstrate realm with moltres
%   - Evaluate the moltres' heat transfer interface and improve on it


\section{Motivation}
The impact of climate change on natural and human systems is increasingly 
apparent \cite{noauthor_climate_2018}.
Increases in global average surface temperatures, sea levels, and larger climate 
extremes are a few consequences brought on by elevated \gls{GHG} concentrations 
\cite{noauthor_climate_2018}.
Energy use and production contribute to two-thirds of the total \gls{GHG}
emissions \cite{noauthor_climate_2018}.
Furthermore, as the human population increases and previously under-developed 
nations rapidly urbanize, global energy demand is forecasted to increase.  
Energy generation technology selection profoundly impacts climate change via 
growing energy demand. 
Large scale deployment of emissions free nuclear power plants could 
significantly reduce GHG production \cite{noauthor_climate_2018}.  
However, large scale nuclear power deployment faces challenges of cost and 
safety \cite{petti_future_2018}. 
The nuclear power industry must overcome these challenges to ensure continued 
global use and expansion of nuclear energy technology to provide low-carbon 
electricity in the global effort to battle climate change. 

In an effort to enhance the role of nuclear energy in our global energy 
eco-system, the Generation IV International Forum was created to lead and plan 
the research and development roadmap to support a new generation IV of innovative 
nuclear energy systems \cite{gif_technology_2002}.
The goals of Generation IV nuclear systems are defined in four areas: 
sustainability, economics, safety and reliability, and proliferation resistance 
and physical protection \cite{gif_technology_2002}. 
Table \ref{tab:goals-gen4} summarizes the goals in each area. 

\begin{table}[]
    \centering
    \onehalfspacing
    \caption{Goals of Generation IV Nuclear Systems \cite{gif_technology_2002,
    behar_technology_2014}}
	\label{tab:goals-gen4}
    \small
    \begin{tabular}{l|l}
    \hline
                               \textbf{Area} & \textbf{Goals} \\ \hline
    Sustainability   & - Have a positive impact on the environment through displacement of \\
    & polluting energy and transportation sources by nuclear electricity generation \\
    & and nuclear-produced hydrogen \\ 
    & - Promote long-term availability of nuclear fuel \\
    & - Minimize volume, lifetime and toxicity of nuclear waste \\ \hline
    Economics & - Have a life cycle and energy production cost advantage over other energy \\
    & sources \\ 
    & - Reduce economic risk to nuclear projects by developing plants using \\
    & innovative fabrication and construction techniques \\ \hline
    Safety and Reliability   & - Increase the use of inherent safety features, robust designs, and \\
    & transparent safety features that can be understood by non experts \\ 
    & - Enhance public confidence in the safety of nuclear energy \\\hline
    Proliferation Resistance & - Provide continued effective proliferation resistance of nuclear energy \\
    and Physical Protection & systems through improved design features and other measures \\ 
    & - Increase robustness of new facilities \\ \hline
    \end{tabular}
    \end{table}

Based on the goals, an evaluation and selection methodology was developed which 
culminated in the selection of six Generation IV systems: \gls{GFR}, 
\gls{LFR}, \gls{MSR}, \gls{SFR}, \gls{SCWR}, and \gls{VHTR} \cite{gif_technology_2002}. 
The reactor systems of interest in this dissertation are the \gls{MSR} and \gls{VHTR} systems. 
The \gls{MSR} system produces fission power in a circulating molten salt 
fuel mixture and has a closed fuel cycle tailored to the efficient 
utilization of plutonium and minor actinides. 
Molten fluoride salts have very low vapor pressure which reduces stress on the 
system and there is also inherent system safety due to fail-safe drainage, passive 
cooling, and a low inventory of volatile fission products in the fuel.  
The \gls{MSR} system is top-ranked in sustainability because of its closed 
fuel cycle and excellent performance in waste burndown. 
It is rated good in safety, and in proliferation resistance and physical 
protection due to its inherent safety features, and its rated neutral in economics 
because of its large number of subsystems \cite{gif_technology_2002}.  
The \gls{VHTR} system has a once-through uranium cycle and is primarily aimed 
at use with high-temperature process heat applications, such as hydrogen 
production. 
It is a graphite-moderated, helium-cooled reactor that uses \gls{TRISO} fuel 
which does not degrade at high burnup and temperature.  
The \gls{VHTR} system is highly ranked in economics because of its high hydrogen 
production efficiency, and in safety and reliability because of the inherent 
safety features of the fuel and reactor. 
It is rated good in proliferation resistance and physical protection, and 
neutral in sustainability because of its open fuel cycle \cite{gif_technology_2002}. 

% maybe need a better connection from MSR VHTR to FHR? 
In this dissertation, we will be exploring the \gls{FHR} reactor concept, which 
is a combination of the best aspects of \gls{MSR} and \gls{VHTR} technologies. 
The \gls{FHR} uses high-temperature coated-particle fuel (similar to the \gls{VHTR}) 
and a low pressure liquid fluoride-salt coolant (similar to the \gls{MSR})
\cite{forsberg_fluoride-salt-cooled_2012,facilitators_fluoride-salt-cooled_2013}.

In recent years, additive manufacturing technology has advanced and  
altered the way in which components are designed and manufactured 
\cite{simpson_considerations_2019}. 
Key additive manufacturing technologies relevant to nuclear reactor core 
structures are \gls{SLM}, \gls{EBM}, and \gls{L-DED}. 
The \gls{SLM} and \gls{EBM} techniques have been used for automotive and aircraft component 
fabrication \cite{murr_frontiers_2016}.  
Successful examples of additive manufacturing applied in the the aircraft industry 
are Boeing’s use of additive manufacturing to reduce weight in the 787 Dreamliner
\cite{noauthor_printed_2017} and SES-15 spacecraft \cite{noauthor_boeing_nodate}. 
Similar to the nuclear industry, the aerospace industry is highly regulated, thus 
successful applications of additive manufacturing to the aerospace industry 
is promising for the nuclear industry.  
Using additive manufacturing to fabricate nuclear reactor components will 
drastically reduce cost, timelines, increase safety and performance by 
tailoring local material properties and redesigning geometries for optimal load paths 
\cite{simpson_considerations_2019}. 

With further advancement of these additive manufacturing technologies, a reactor 
core could be 3D printed in the near future. 
\gls{ORNL} is leading this initiative through the 2019 \gls{TCR} Demonstration 
Program. 
The \gls{TCR} program will leverage recent scientific achievements in advanced 
manufacturing, nuclear materials, machine learning, and computational modeling 
and simulation to build a microreactor. 
The program targets to design, manufacture, and operate a demonstration reactor 
by 2023 \cite{terrani_transformational_2019}. 
% explore funky designs :D 
Applying additive manufacturing to nuclear reactor design will enable complex 
reactor geometries that are not limited by previous manufacturing constraints. 
This opens the door for re-examination of nuclear reactor optimization 
\cite{sobes_artificial_2020}. 
Optimization efforts towards classically-manufactured nuclear reactors and now
3D printed nuclear reactors have focused on uniform shapes such as radius of the 
core, height of cylinder, enrichment of fuel, etc 
\cite{sobes_artificial_2020,sacco_two_2006,kumar_new_2015,pereira_parallel_2008}. 
Leveraging additive manufacturing technology enables us to surpass classical 
manufacturing constraints such as straight fuel channels or homogenous fuel enrichment, 
and look forwards to optimize for non-uniform channel shapes, inhomogeneous 
fuel enrichment throughout the core, etc. 

Multi-objective design problems inevitably require a trade off between 
desirable attributes \cite{byrne_evolving_2014,simon_sciences_2019}. 
In nuclear reactor design there are many trade offs, one example is the 
trade-off between neutron economy and fuel enrichment. 
A reactor design must have sufficient neutron economy to ensure criticality, 
but must also have a low fuel enrichment to reduce proliferation risk.
Conflicting objectives means that there is no one perfect solution, but a set
of equally optimal solutions \cite{byrne_evolving_2014}.
Multi-objective problems are difficult to optimize, such problems 
cannot be handled by classical optimization methods such as gradient 
methods, because only the local optimum will be found \cite{renner_genetic_2003}. 
Evolutionary algorithms have proven to be successful 
methods to optimize multi-objective problems \cite{krish_practical_2011} as 
they can find a solution near the global optimum \cite{renner_genetic_2003}
and they also take advantage of parallel systems for reduced computational 
cost. 
The most popular evolutionary algorithms used to solve multi-objective 
problems are genetic algorithms 
\cite{byrne_evolving_2014, krish_practical_2011}. 
Genetic algorithms imitate natural selection to evolve solutions 
by (1) maintaining a population of solutions, (2) allowing 
fitter solutions reproduce, and (3) letting lesser fit solutions die off, 
resulting in final solutions that are better than the previous generations 
\cite{renner_genetic_2003}. 
Thus, we will pursue the evolutionary algorithm optimization technique 
in this work. 

Therefore in this work, we propose the design of an optimization tool that uses 
the evolutionary algorithm optimization technique with nuclear transport and 
thermal hydraulics software to explore non-uniform \gls{FHR} reactor core parameters 
in an effort to further optimize reactor systems and fully take advantage of 
additive manufacturing technology.  


\section{Objectives}
The main objectives of the proposed work were developed based on leveraging open-source 
artificial intelligence tools with validated open-source nuclear transport and 
thermal hydraulics software to create an open-source tool to easily generate 
optimal reactor designs. 
Accordingly, the objectives are listed below. 

\vspace{0.2cm} 
\noindent
\textbf{Model Fluoride-Salt-Cooled High-Temperature Reactor with established 
nuclear transport and thermal hydraulics software}.
The optimization tool will be applied to the \gls{FHR} concept. 
To demonstrate success in modeling the \gls{FHR} with the nuclear transport and 
thermal hydraulics software prior to use with the optimization tool, I will participate 
in the \gls{OECD} \gls{NEA}'s \gls{FHR} benchmark \cite{noauthor_fluoride_nodate}. 

\vspace{0.2cm} 
\noindent
\textbf{Develop a tool that applies evolutionary algorithms to nuclear 
reactor design optimization}. 
This tool will not re-invent the wheel, it will utilize a well-documented 
and validated open-source evolutionary algorithm python package with established 
nuclear transport and thermal hydraulics software. This tool will run parallel on 
\gls{HPC} machines. This tool will be open-source and follow rules for ensuring 
reproducibility, effectiveness, and usability 
\cite{list_ten_2017,osborne_ten_2014,sandve_ten_2013}. 

\vspace{0.2cm} 
\noindent
\textbf{Demonstrate nuclear reactor design optimization with developed tool 
for a neutronics problem}. 
We will demonstrate successful implementation of the optimization tool with the 
nuclear transport software by optimizing a simple \gls{FHR} model for a single 
objective function. 

\vspace{0.2cm} 
\noindent
\textbf{Demonstrate hyperparameter tuning with developed tool for neutronics problem}.
There are many hyperparameters required for an evolutionary algorithm. 
The hyperparameter selection will impact the effective-ness of the algorithm 
for our problem. 
Therefore, we must conduct hyperparameter tuning to find hyperparameters that work 
best for our problem. 

\vspace{0.2cm} 
\noindent
\textbf{Demonstrate nuclear reactor design optimization and hyperparameter 
search with tool for neutronics and thermal hydraulics problem}.
We will demonstrate successful implementation of the optimization tool and 
hyperparameter tuning with the nuclear transport and thermal hydraulics tools 
for a \gls{FHR} model.  


\section{Outline}
This document outlines the motivation, preliminary work, and future work proposed 
towards developing an open-source optimization tool that applies evolutionary 
algorithms to nuclear transport and thermal hydraulics software to optimize 
nuclear reactor design. 
Chapter 1 provides a description of the motivation and objectives of the 
proposed work. 
Chapter 2 will present a literature review that organizes and reports on previous 
relevant work. 
First it summarizes the history of the \gls{FHR} reactor system and the ongoing 
\gls{OECD}-\gls{NEA} \gls{FHR} benchmark. 
Next, the literature review describes the status of additive manufacturing 
applications in the nuclear industry. 
Then, it focus on types of optimization and previous work towards reactor design
optimization. 
Chapter 3 will describe the \gls{FHR} benchmark specifications and the results 
obtained thus far. 
Chapter 4 will detail the computational design of the REALM python package. 
Chapter 5 will demonstrate nuclear reactor optimization with REALM. 
Chapter 6 will summarize the remaining future work. 
