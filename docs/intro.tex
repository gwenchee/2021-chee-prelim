\chapter{Introduction}
% Main Gist 
% - Reactor design optimization is a highly iterative process which requires
%   back and forth between neutronics and thermal hydraulics designers. 
%   It requires years of experience to have intuition on a global optimum 
%   solution. AI applications can help with this. 
% - I will be applying AI methods to the salt-cooled reactor type because its 
%   an awesome reactor type. 
% Structure 
% - Motivation 
% - Outline of dissertation
%   - FHR Benchmark 
%   - Create a python package to couple genetic algorithm to neutronics / 
%     hydraulics code 
%   - Demonstrate realm with openmc 
%   - Demonstrate realm with moltres
%   - Evaluate the moltres' heat transfer interface and improve on it


\section{Motivation}
The impact of climate change on natural and human systems is increasingly 
apparent \cite{noauthor_climate_2018}.
Increases in global average surface temperatures, sea levels, and larger climate 
extremes are a few consequences brought on by elevated \gls{GHG} concentrations 
\cite{noauthor_climate_2018}.
Energy use and production contribute to two-thirds of the total \gls{GHG}
emissions \cite{noauthor_climate_2018}.
Furthermore, as the human population increases and previously under-developed 
nations rapidly urbanize, global energy demand is forecasted to increase.  
Energy generation technology selection profoundly impacts climate change via 
growing energy demand. 
Large scale deployment of emissions free nuclear power plants could 
significantly reduce GHG production \cite{noauthor_climate_2018}.  

However, large scale nuclear power deployment faces challenges of cost and 
safety \cite{petti_future_2018}. 
The nuclear power industry must overcome these challenges to ensure continued 
global use and expansion of nuclear energy technology. 



\section{Objectives}
This dissertation's objectives were developed based on leveraging open-source 
artificial intelligence tools with validated open-source nuclear transport and 
thermal hydraulics software to create an open-source tool to easily generate 
optimal reactor designs. 
Accordingly, an outline of the steps to accomplish the motivation and goals 
of this dissertation are listed below. 
