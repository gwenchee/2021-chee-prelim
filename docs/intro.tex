\chapter{Introduction}
\label{chap:intro}
% Main Gist 
% - Reactor design optimization is a highly iterative process which requires
%   back and forth between neutronics and thermal hydraulics designers. 
%   It requires years of experience to have intuition on a global optimum 
%   solution. AI applications can help with this. 
% - I will be applying AI methods to the salt-cooled reactor type because its 
%   an awesome reactor type. 
% Structure 
% - Motivation 
% - Outline of dissertation
%   - FHR Benchmark 
%   - Create a python package to couple genetic algorithm to neutronics / 
%     hydraulics code 
%   - Demonstrate realm with openmc 
%   - Demonstrate realm with moltres
%   - Evaluate the moltres' heat transfer interface and improve on it


\section{Motivation}
Increased global average surface temperatures, sea levels, and severe weather 
events caused by elevated \gls{GHG} concentrations show the substantial 
negative impact of climate change on natural and human systems
\cite{noauthor_climate_2018}.
Energy use and production contribute two-thirds of total \gls{GHG}
emissions \cite{noauthor_climate_2018}.
Furthermore, as the human population increases and previously underdeveloped 
nations rapidly industrialize, global energy demand will continue to rise.  
Because energy generation technology selection profoundly impacts climate change, 
large scale emissions-free nuclear power deployment could 
significantly reduce GHG production but faces both cost and perceived adverse 
safety challenges \cite{noauthor_climate_2018, petti_future_2018}. 
The nuclear power industry must overcome these challenges to ensure continued 
global use and expansion of nuclear energy technology to provide low-carbon 
electricity worldwide.
The Generation IV International Forum formation aims to enhance the role of 
nuclear energy in our global energy ecosystem by leading and planning 
research and development to support a new and innovative Generation IV
nuclear energy systems \cite{gif_technology_2002}.
Generation IV nuclear systems' target goals in four areas: sustainability, 
economics, safety and reliability, and proliferation resistance and physical 
protection \cite{gif_technology_2002}. 
Table \ref{tab:goals-gen4} summarizes the goals in each area. 

\begin{table}[]
    \centering
    \onehalfspacing
    \caption{Goals of Generation IV Nuclear Systems \cite{gif_technology_2002,
    behar_technology_2014}}
	\label{tab:goals-gen4}
    \footnotesize
    \begin{tabular}{l|l}
    \hline
                               \textbf{Area} & \textbf{Goals} \\ \hline
    Sustainability   & - Have a positive impact on the environment through the displacement of \\
    & polluting energy and transportation sources by nuclear electricity generation \\
    & and nuclear-produced hydrogen \\ 
    & - Promote long-term availability of nuclear fuel \\
    & - Minimize volume, lifetime, and toxicity of nuclear waste \\ \hline
    Economics & - Have a life cycle and energy production cost advantage over other energy \\
    & sources \\ 
    & - Reduce economic risk to nuclear projects by developing plants using \\
    & innovative fabrication and construction techniques \\ \hline
    Safety and Reliability   & - Increase the use of robust designs, and inherent and transparent safety\\
    & features that can be understood by non-experts \\ 
    & - Enhance public confidence in the safety of nuclear energy \\\hline
    Proliferation Resistance & - Provide continued effective proliferation resistance of nuclear energy \\
    and Physical Protection & systems through improved design features and other measures \\ 
    & - Increase the robustness of new facilities \\ \hline
    \end{tabular}
    \end{table}

The Generation IV International Forum's methodology working groups developed 
an evaluation and selection methodology based on these goals 
and correspondingly selected six Generation IV systems: \glspl{GFR}, 
\glspl{LFR}, \glspl{MSR}, \glspl{SFR}, \glspl{SCWR}, and \glspl{VHTR} 
\cite{gif_technology_2002}. 
This proposed work will consider the \gls{MSR} and \gls{VHTR} systems. 

\gls{MSR} systems produce fission power in a circulating molten salt fuel mixture. 
It has a closed fuel cycle tailored to the efficient utilization of plutonium 
and minor actinides. 
Molten fluoride salts have very low vapor pressure, which reduces stress on the 
system. 
MSR systems also incorporate inherent system safety with fail-safe drainage, 
passive cooling, and a low inventory of volatile fission products in the fuel. 
\gls{MSR} systems closed fuel cycle and excellent waste burndown performance 
make it top-ranked in sustainability. 
MSR systems are top-ranked in sustainability because of their closed fuel cycle 
and excellent waste burndown performance. 
They rate well in safety and proliferation resistance and physical protection, 
due to their inherent safety features, and rate neutrally in economics because of
their large number of subsystems \cite{gif_technology_2002}.  

\gls{VHTR} systems use a once-through uranium cycle and leverage their 
high outlet temperature for high-temperature heat applications, such as hydrogen 
production. 
Graphite-moderated and helium-cooled, \glspl{VHTR} use \gls{TRISO} fuel
which easily withstands high burnup and temperature.  
\gls{VHTR} systems' high hydrogen production efficiency, high safety and 
reliability, and inherent fuel and reactor safety features make it highly 
ranked in economics. 
An open fuel cycle results in \gls{VHTR} systems' ranking well in proliferation 
resistance and physical protection and neutrally in sustainability 
\cite{gif_technology_2002}. 

In the proposed work, we explore the \glspl{FHR} concept, which 
combines the best aspects of \gls{MSR} and \gls{VHTR} technologies. 
\glspl{FHR} use high-temperature coated-particle fuel (similar to the \glspl{VHTR}) 
and a low-pressure liquid fluoride-salt coolant (similar to the \glspl{MSR})
\cite{forsberg_fluoride-salt-cooled_2012,facilitators_fluoride-salt-cooled_2013}.
% add more connecction

In recent years, additive manufacturing technology, popularly known as `3D printing', 
has advanced and altered the manufacturing and design of engineered hardware
\cite{simpson_considerations_2019}. 
The automotive and aircraft industries have successfully fabricated car and 
airplane components with key additive manufacturing technologies that are 
relevant to nuclear reactor core structures \cite{murr_frontiers_2016}.  
For example, Boeing successfully used additive manufacturing to reduce weight 
in the 787 Dreamliner \cite{noauthor_printed_2017} and SES-15 spacecraft 
\cite{noauthor_boeing_nodate}. 
The highly-regulated aerospace industry's successful additive manufacturing 
applications shows promise for the also highly-regulated nuclear industry.  
Using additive manufacturing to fabricate nuclear reactor components could 
drastically reduce cost and timelines, and increase safety and performance by 
tailoring local material properties and redesigning for optimal geometries
\cite{simpson_considerations_2019}. 

With further advancement of additive manufacturing technologies, a reactor 
core could be 3D printed in the near future. 
\gls{ORNL} leads this initiative through the 2019 \gls{TCR} Demonstration 
Program. 
The \gls{TCR} program will leverage recent scientific achievements in advanced 
manufacturing, nuclear materials, machine learning, and computational modeling 
and simulation to build a microreactor. 
The program aims to design, manufacture, and operate a demonstration reactor 
by 2023 \cite{terrani_transformational_2019}. 
% explore funky designs :D 
Applying additive manufacturing to nuclear reactor design will free complex 
reactor geometries from previous manufacturing constraints,
opening the door for a re-examination of nuclear reactor optimization 
\cite{sobes_artificial_2020}. 
Optimization efforts towards classically manufactured nuclear reactors, and now
3D printed nuclear reactors have focused on parameters such as core radius, 
cylinder height, fuel enrichment, etc 
\cite{sobes_artificial_2020,sacco_two_2006,kumar_new_2015,pereira_parallel_2008}. 
Leveraging additive manufacturing technology enables us to surpass classical 
manufacturing constraints, such as straight fuel channels or homogenous fuel 
enrichment, and optimize for arbitrary geometries and parameters 
such as non-uniform channel shapes, and inhomogeneous fuel distribution throughout 
the core. 

Multi-objective design problems inevitably require a trade-off between 
desirable attributes \cite{byrne_evolving_2014,simon_sciences_2019}. 
For example, the neutron economy and fuel enrichment trade-off in nuclear 
reactor design. 
A reactor design should maximize neutron economy and minimize fuel enrichment 
to reduce proliferation risk and cost. 
Conflicting objectives results in no \textit{one} perfect solution but 
a \textit{set} of equally optimal solutions \cite{byrne_evolving_2014}.
Multi-objective problems are challenging to optimize; therefore, they cannot be 
handled by classical optimization methods, such as gradient methods, which may 
find local optima while missing the global optimum \cite{renner_genetic_2003}. 
Evolutionary algorithms have proven successful in optimizing multi-objective problems 
\cite{krish_practical_2011}, as they can find solutions at the global 
optimum \cite{renner_genetic_2003}.
They also take advantage of parallel systems for reduced computational cost.
Genetic algorithms are the most frequently used evolutionary algorithms for 
solving multi-objective problems \cite{byrne_evolving_2014, krish_practical_2011}.
Genetic algorithms imitate natural selection to evolve solutions by 
\cite{renner_genetic_2003}: 
\begin{enumerate}
    \item maintaining a population of solutions
    \item allowing fitter solutions to reproduce
    \item letting lesser fit solutions die off, resulting in final solutions that are 
    better than the previous generations
\end{enumerate} 

Therefore, in this work, I propose designing an optimization tool that uses 
the evolutionary algorithm optimization technique with nuclear transport and 
thermal-hydraulics software. 
With this tool, I will explore nonuniform FHR reactor core parameters, now 
possible with additive manufacturing technology, to fully optimize reactor systems. 

\section{Objectives}
I developed the proposed work's main objectives based on leveraging 
artificial intelligence tools with validated nuclear transport and 
thermal-hydraulics software to create an open-source tool which 
generates optimal reactor designs quickly. 
The proposed work's objectives are: 

\vspace{0.2cm} 
\noindent
\begin{enumerate}[label=\textbf{\Roman*}]
\item \textbf{Model the \gls{FHR} with established 
nuclear transport and thermal-hydraulics software}.
To demonstrate success in modeling the \gls{FHR} with nuclear transport and 
thermal-hydraulics software before using the optimization tool, we will 
participate in the \gls{OECD} \gls{NEA}'s \gls{FHR} benchmark 
\cite{noauthor_fluoride_nodate}. 

\item \textbf{Develop a tool that applies evolutionary algorithms to optimize nuclear 
reactor design}. 
This tool will not reinvent the wheel---it will utilize a well-documented 
and validated open-source evolutionary algorithm Python package with established 
nuclear transport and thermal-hydraulics software. This tool will run parallel on 
\gls{HPC} machines, be open-source, and follow the rules for ensuring 
reproducibility, effectiveness, and usability 
\cite{list_ten_2017,osborne_ten_2014,sandve_ten_2013}. 

\item \textbf{Optimize a nuclear reactor design with the optimization tool and a 
neutronics software}. 
We will demonstrate successful implementation of the optimization tool with a
nuclear transport software by optimizing a simple \gls{FHR} model for a single 
objective function. 

\item \textbf{Tune hyperparameters with the optimization tool for a neutronics problem}.
Hyperparameter selection will impact the effectiveness of the algorithm 
for our problem. 
Therefore, we must conduct a hyperparameter search to find ones that work best 
for our problem. 

\item \textbf{Demonstrate nuclear reactor design optimization and hyperparameter 
search with the optimization tool for a neutronics and thermal-hydraulics problem}.
We will demonstrate successful implementation of the optimization tool and 
hyperparameter tuning with the nuclear transport and thermal-hydraulics tools 
for an \gls{FHR} model.  
\end{enumerate}

\section{Outline}
This document outlines the motivation, preliminary work, and future work proposed 
toward developing an open-source reactor evolutionary algorithm optimization tool 
to optimize nuclear reactor design beyond classical parameters. 
Chapter 1 describes the motivation and objectives of the proposed work. 
Chapter 2 presents a literature review that organizes and reports on previous 
relevant work. 
Chapter 3 describes the \gls{FHR} benchmark specifications and the results 
obtained thus far. 
Chapter 4 details the computational design of the developed optimization 
tool. 
Chapter 5 demonstrates nuclear reactor optimization with the optimization 
tool.  
Chapter 6 summarizes the remaining future work. 
