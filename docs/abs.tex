\vspace{-1.5cm}
The nuclear power industry must overcome cost and safety challenges to ensure 
continued global use and expansion of nuclear energy technology to provide 
low-carbon electricity worldwide.
The Generation IV International Forum identified six nuclear reactor systems 
that promise significant advances in safety, sustainability, efficiency, 
and cost over existing designs.
The \acrfull{FHR} system, specifically the \acrfull{AHTR} design, combines the 
best aspects of two identified Generation IV systems: \acrfull{MSR} and \acrfull{VHTR}. 
The \acrshort{AHTR} uses the \acrshort{MSR}'s low-pressure liquid fluoride-salt 
coolant and \acrshort{VHTR}'s high-temperature coated-particle fuel. 
The \acrshort{AHTR}'s fuel geometry has triple heterogeneity, resulting in complex 
reactor physics and significant modeling challenges. 
To further understand and address the technical challenges associated with the 
\acrshort{AHTR} design, this work proposes participation in the \acrlong{OECD}-\acrlong{NEA}'s 
\acrshort{FHR} benchmarking exercise by modeling its neutronics and thermal-hydraulics 
with OpenMC and Moltres nuclear software. 
In the proposed work, I will also explore the impact of additive manufacturing
technology advancements on reactor geometry optimization, specifically for the 
\acrshort{AHTR} design.
Leveraging additive manufacturing technology enables us to surpass classical manufacturing
constraints, such as straight fuel channels or homogenous fuel enrichment, and optimize for
arbitrary geometries and parameters such as non-uniform channel shapes and 
inhomogeneous fuel distribution throughout the core.
This work proposes to design and demonstrate an optimization tool that uses the 
evolutionary algorithm optimization technique with neutron transport and 
thermal-hydraulics software to find new optimal reactor geometries enabled by 
additive manufacturing technology.