\chapter{Literature Review}
% Main Gist 
% - History of FHR reactor type and why there is a renewed interest in this
%   reactor type (start ups, labs etc.)
% - Past applications of AI to nuclear reactor design. 
% Structure 
% - Fluoride-Salt-Cooled High-Temperature Reactor (why salt cooled triso fuelled 
%   reactors are cool)
%   - FHR Modeling Challenges (why benchmark exists)
%   - Description of benchmark
% - Artificial Intelligence Applications to Nuclear Reactor Design Optimization
%   - Past work 
%   - Classical vs Evolutionary methods 

\section{Fluoride-Salt-Cooled High-Temperature Reactor}
The \gls{FHR} is a reactor concept introduced in 2012 that uses high-temperature 
coated-particle fuel and a low pressure liquid fluoride-salt coolant 
\cite{forsberg_fluoride-salt-cooled_2012,facilitators_fluoride-salt-cooled_2013}.
\gls{FHR} technology combines the best aspects of \gls{MSR} and \gls{VHTR} 
(or \gls{HTGR}) technologies. 
Molten fluoride salts as working fluids for nuclear reactors have been explored 
since the 1960s and are desirable because of the salts' high-temperature 
performance and overall chemical stability \cite{scarlat_design_2014}.  
Using molten salts for reactor coolant introduces inherent safety compared 
to water due to the salts' high boiling temperature and high volumetric 
heat capacity, eliminating the risk of coolant boiling off, resulting in 
fuel elements overheating \cite{ho_molten_2013}. 
The leading candidate coolant salt is the fluoride salt Li$_2$BeF$_4$ (FLiBe), 
which remains liquid without pressurization up to 1400 $^{\circ}$C and a larger 
$\rho C_p$ than water \cite{ho_molten_2013,forsberg_fluoride-salt-cooled_2012}. 
\glspl{FHR} are favorable compared to a liquid fuel reactor, such as
\gls{MSR} systems, because the solid fuel cladding adds an extra barrier to fission 
product release 
\cite{ho_molten_2013}.

\gls{VHTR} technology has been studied since the 1970s because it delivers 
heat at substantially high temperatures than \glspl{LWR} resulting in 
the following advantages: increased power conversion efficiency, reduced 
waste heat generation, and co-generation and process heat capabilities 
\cite{scarlat_design_2014}. 
In \glspl{VHTR}, the helium coolant is held at a high pressure of approximately 
100 atm, whereas the \gls{FHR}'s FLiBe coolant is at room pressure, resulting in lower 
construction costs since a thick concrete reactor vessel is not required.
The molten salt coolant has superior cooling and moderating properties compared 
to helium coolant in \glspl{VHTR}, resulting in \glspl{FHR} operating at 
power densities two to six times higher than  \glspl{VHTR} 
\cite{scarlat_design_2014,forsberg_fluoride-salt-cooled_2012}.
Therefore, by combining the FLiBE coolant from \gls{MSR} technology and 
\gls{TRISO} particles from \gls{VHTR} technology, the \gls{FHR} benefits from 
the low operating pressure and large thermal margin provided by using a molten 
salt coolant and the accident-tolerant qualities of \gls{TRISO} particle fuel. 

There are several types of \gls{FHR} conceptual designs that are being developed 
worldwide: \gls{PBFHR} at UCB \cite{scarlat_current_2014,krumwiede_three-dimensional_2013}, 
the large central-station \gls{AHTR} at \gls{ORNL} \cite{holcomb_core_2011, varma_ahtr_2012}, 
the \gls{SmAHTR} at ORNL \cite{greene_pre-conceptual_2010}, and the \gls{SF-TMSR}
at the \gls{SINAP} in China \cite{liu_preliminary_2016}. 
% More descriptions about these different FHR designs. ?? RAMEY one?

\subsection{FHR Modeling Challenges}

\subsection{FHR Benchmark}

\section{Artificial Intelligence Applications to Nuclear Reactor Design}