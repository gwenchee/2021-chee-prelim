\chapter{Literature Review}
\label{chap:lit-review}
% Main Gist 
% - History of FHR reactor type and why there is a renewed interest in this
%   reactor type (start ups, labs etc.)
% - Past applications of AI to nuclear reactor design. 
% Structure 
% - Fluoride-Salt-Cooled High-Temperature Reactor (why salt cooled triso fuelled 
%   reactors are cool)
%   - FHR Modeling Challenges (why benchmark exists)
%   - Description of benchmark
% - Artificial Intelligence Applications to Nuclear Reactor Design Optimization
%   - Past work 
%   - Classical vs Evolutionary methods 

This chapter provides a literature review of relevant past research efforts 
that give context to this proposed work.
Recent advancements in \gls{AM} applications for nuclear reactor core 
components has removed traditional manufacturing constraints on reactor design, 
enabling reactor designers to reexamine optimization.
In the proposed work, we aim to apply evolutionary algorithm methods to explore 
non-conventional reactor geometries and fuel distributions, to apply a fresh 
perspective to nuclear reactor optimization. 
With growing interest in the nuclear science community and benefits of the 
\gls{FHR}, we chose to apply the optimization methods to this reactor type, 
and also participate in the \gls{OECD} \gls{NEA}'s \gls{FHR} benchmarking exercise. 
Thus, we begin this literature review with an overview of the \gls{FHR} concept, 
then go into detail about one specific \gls{FHR} design: the \gls{AHTR}, 
previous efforts and technical challenges of modeling the design, and a 
description of how these efforts led to the \gls{OECD} \gls{NEA}'s initiation of 
the \gls{AHTR} benchmark.
Next, we outline \gls{AM} history and describe the current research towards applying 
\gls{AM} to the fabrication of nuclear reactor core components. 
We also review previous efforts towards nuclear reactor design optimization, 
describe how \gls{AM} of nuclear reactor components enables optimization for 
less constrained reactor geometries, and types of optimization methods that can be 
leveraged in a large less-constrained design space, such as evolutionary algorithms.  
Finally, we give a background of evolutionary algorithms, go into detail on a 
specific evolutionary algorithm: the genetic algorithm, and how it works to robustly conduct global 
optimization. 

\section{Fluoride-Salt-Cooled High-Temperature Reactor}
\label{sec:fhr}
The \gls{FHR} is a reactor concept introduced in 2012 that uses high-temperature 
coated-particle fuel and a low-pressure liquid fluoride-salt coolant 
\cite{forsberg_fluoride-salt-cooled_2012,facilitators_fluoride-salt-cooled_2013}.
\gls{FHR} technology combines the best aspects of \gls{MSR} and \gls{VHTR} 
(or \gls{HTGR}) technologies. 
Molten fluoride salts as working fluids for nuclear reactors have been explored 
since the 1960s and are desirable because of their high-temperature 
performance and overall chemical stability \cite{scarlat_design_2014}.  
Using molten salts for reactor coolant introduces inherent safety compared 
to water due to the salts' high boiling temperature and high volumetric 
heat capacity.
This eliminates the risk of coolant boiling off, resulting in fuel elements 
overheating \cite{ho_molten_2013}. 
The leading candidate coolant salt is the fluoride salt Li$_2$BeF$_4$ (FLiBe), 
which remains liquid without pressurization up to 1400 $^{\circ}$C and has a larger 
heat capacity than water \cite{ho_molten_2013,forsberg_fluoride-salt-cooled_2012}. 
\glspl{FHR} are favorable compared to liquid fuelled reactors, such as
\gls{MSR} systems, because the \gls{TRISO} particles' solid fuel cladding adds 
an extra barrier to fission product release \cite{ho_molten_2013}.

\gls{VHTR} technology has been studied since the 1970s because it delivers 
heat at substantially higher temperatures than \glspl{LWR}, resulting in 
the following advantages: increased power conversion efficiency, reduced 
waste heat generation, and co-generation and process heat capabilities 
\cite{scarlat_design_2014}. 
In \glspl{VHTR}, the helium coolant is held at a high pressure of approximately 
100 atm, whereas the \gls{FHR}'s FLiBe coolant is at room pressure, resulting in lower 
construction costs since a thick concrete reactor vessel is not required.
The molten salt coolant has superior cooling and moderating properties compared 
to helium coolant in \glspl{VHTR}, resulting in \glspl{FHR} operating at 
power densities two to six times higher than  \glspl{VHTR} 
\cite{scarlat_design_2014,forsberg_fluoride-salt-cooled_2012}.
Therefore, by combining the FLiBE coolant from \gls{MSR} technology and 
\gls{TRISO} particles from \gls{VHTR} technology, the \gls{FHR} benefits from 
the low operating pressure and large thermal margin provided by using a molten 
salt coolant and the accident-tolerant qualities of \gls{TRISO} particle fuel. 

Several types of \gls{FHR} conceptual designs exist worldwide: \gls{PBFHR} at 
\gls{UCB} with circulating pebble-fuel 
\cite{scarlat_current_2014,krumwiede_three-dimensional_2013}, the \gls{SF-TMSR} 
at the \gls{SINAP} in China with static pebble-fuel \cite{liu_preliminary_2016}, 
the large central-station \gls{AHTR} at \gls{ORNL} \cite{holcomb_core_2011, varma_ahtr_2012} and 
the \gls{SmAHTR} at ORNL \cite{greene_pre-conceptual_2010} with static plate-fuel. 

\subsection{\acrlong{AHTR} Design}
This proposed work focuses on the \gls{FHR} design with hexagonal fuel assemblies
consisting of \gls{TRISO} fuel particles embedded in planks, i.e., the 
\gls{AHTR} design developed by ORNL. 
The \gls{AHTR} has 3400 MWt thermal power and 1400 MW electric power with
inlet/outlet temperatures of 650/700$^{\circ}$C \cite{varma_ahtr_2012}.  
Figure \ref{fig:ahtr} shows the prismatic AHTR's fuel assembly and core 
configuration.  
Each hexagonal fuel element features plate-type fuel consisting of eighteen planks 
arranged in three diamond-shaped sectors, with a central Y-shaped structure 
and external channel (wrapper).
Each fuel plank contains an isostatically pressed carbon with fuel stripes 
on each plank's outer side.
The fuel stripes are prismatic regions composed of a graphite matrix filled with 
a cubic lattice of \gls{TRISO} particles. 
The core consists of 252 assemblies radially surrounded by reflectors
\cite{ramey_monte_2018}. 
Chapter \ref{chap:fhr-benchmark} details the specifications of the AHTR geometry
modeled in this proposed work.

\begin{figure}[]
    \centering
    \includegraphics[width=\linewidth]{ahtr.png} 
    \caption{\acrlong{AHTR} fuel assembly (left) and core configuration (right) 
    \cite{ramey_monte_2018}.}
    \label{fig:ahtr}
\end{figure}

\subsection{Previous AHTR modeling efforts and challenges}
Modeling and simulation of the \gls{AHTR} design have been an ongoing effort 
since its conception in 2003 \cite{forsberg_molten-salt-cooled_2003}. 
The \gls{AHTR} core design differs significantly from the present \gls{LWR}-based 
nuclear power plants. 
These differences lead to modeling challenges and the need to verify and 
validate modeling and simulation methods \cite{ramey_monte_2018}. 
Verification and validation of neutronics and thermal-hydraulics tools' 
capability to successfully model the \gls{AHTR} design are crucial steps 
in support of licensure of the \gls{AHTR} design towards the eventual goal 
of deployment \cite{rahnema_phenomena_2019,rahnema_current_2015}. 
Several neutronic studies conducted along the way to the current \gls{AHTR} 
design have shed light on the technical challenges facing the design 
\cite{ramey_monte_2018,holcomb_fluoride_2013,greene_pre-conceptual_2010}. 


\gls{Georgia Tech} led an Integrated Research Project to 
understand challenges in \gls{AHTR} materials and modeling its neutronics and 
thermal-hydraulics \cite{zhang_integrated_2019}. 
During the research project, a panel of subject matter experts 
generated a \gls{PIRT}.
The \gls{PIRT} identifies areas that need additional research to better 
understand important phenomena for adequate future modeling
\cite{rahnema_phenomena_2019}. 
Table \ref{tab:phenomena} lists the phenomena identified as requiring further 
research. 

\begin{table}[]
    \centering
    \onehalfspacing
    \caption{\acrlong{PIRT} identified \acrlong{AHTR} physical phenomena requiring 
    further research \cite{rahnema_phenomena_2019}.}
	\label{tab:phenomena}
    \small
    \begin{tabular}{l|l}
    \hline
    \textbf{Category} & \textbf{Phenomena} \\ \hline
    Fundamental cross section data & - Moderation in FliBe \\
    & - Thermalization in FliBe \\
    & - Absorption in FliBe \\
    & - Thermalization in carbon \\
    & - Absorption in carbon \\ \hline
    Material Composition & - Fuel particle distribution \\ \hline
    Computational Methodology & - Solution Convergence \\ 
    & - Granularity of depletion regions \\
    & - Multiple heterogeneity treatment for generating multigroup \\ 
    & cross sections \\
    & - Selection of multigroup structure \\
    & - Boundary conditions for multigroup cross section generation \\ \hline 
    General Depletion & Spectral history \\ \hline 
    \end{tabular}
    \end{table}

The \gls{AHTR} has a complex core design due to the multiple heterogeneity 
present in the fuel introduced by \gls{TRISO} particles' presence embedded in 
planks \cite{ramey_monte_2018,rahnema_phenomena_2019}.
We must obtain detailed reference power distributions to assess lower-fidelity 
models' accuracy, thus, requiring accurate models of the \gls{AHTR}'s complex 
geometry with individual \gls{TRISO} particle fidelity.
However, it is challenging, particularly for deterministic codes that
use multigroup cross sections and traditional homogenization methods
\cite{ramey_monte_2018}, which are insufficient to capture the correct physics 
in \glspl{AHTR} due to the multiple heterogeneity \cite{ramey_monte_2018}. 
In the \gls{AHTR}, single and multiple slab homogenization decreased computation time 
by an order of 10; however, they introduce a nontrivial error of $\sim$3\%
\cite{ramey_monte_2018,cisneros_neutronics_2012}.
To determine the feasibility and safety of the \gls{AHTR} design, we must 
calculate core physics parameters to an acceptable uncertainty. 
For Monte Carlo codes, increasing neutron histories reduces statistical 
uncertainty but comes at an increased computational cost, requiring the use of 
supercomputers to run the simulations.

Another technical challenge the \gls{AHTR} design faces is the uncertainty of 
the graphite and carbonaceous moderator material properties: densities, 
temperatures, and thermal scattering data.
Problematically, the thermal scattering data ($S(\alpha,\beta)$ matrices) for 
the bound nuclei in the \gls{FLiBe} salt are lacking \cite{ramey_monte_2018}. 
Mei et al. \cite{mei_investigation_2013} and Zhu et al. \cite{zhu_thermal_2017} 
examined the thermal scattering behavior of solid and liquid \gls{FLiBe}.
They concluded that the bound and free atom cross section of \gls{FLiBe} are 
identical above 0.1eV and diverges below 0.01eV, which means that the use or 
absence of thermal scattering data will impact the accuracy of the results 
\cite{ramey_monte_2018}. 

\subsection{AHTR Benchmark}
To address and further understand the technical challenges described 
in the previous section, in 2019, the OECD-NEA initiated a benchmark to assess the 
modeling and simulation capabilities for the \glspl{AHTR} with 
\gls{TRISO} fuel embedded in fuel planks of hexagonal fuel elements
\cite{noauthor_fluoride_nodate}. 
The benchmark plans to have three phases, starting from a single fuel element 
simulation without burnup, gradually extending to full core depletion and feedback. 
The benchmark's overarching objective is to identify the applicability, accuracy, 
and practicality of the latest methods and codes to assess the current state 
of the art of FHR simulation and modeling \cite{petrovic_preliminary_2021}. 
The benchmark also enables the cross-verification of codes and methodologies 
for the challenging \gls{AHTR} geometry, which is especially useful since 
applicable reactor physics experiments for code validation are scarce 
\cite{petrovic_fhrahtr_2019,petrovic_preliminary_2021}. 
Chapter \ref{chap:fhr-benchmark} will provide a detailed description of the 
benchmark phases and results obtained so far.

\section{Additive Manufacturing}
\acrlong{AM} (AM) is the formalized term for what used to be called `rapid prototyping' 
and what is popularly called `3D printing' \cite{gibson_additive_2014}. 
The basic principle of \gls{AM} is that a model is initially generated using a
\gls{3D CAD} system and then fabricated directly without the need for process 
planning. 
In \gls{AM}, the parts are made by adding materials in layers; each layer is a 
thin cross section of the \gls{3D CAD} designed part, as opposed 
to subtractive manufacturing methods such as traditional machining
\cite{standard_standard_2012}. 
All commercialized \gls{AM} machines to date use a layer-based approach, and 
the major ways that they differ are in materials, layer creation method, and 
how the layers are bonded to each other \cite{gibson_additive_2014}.
These major differences will determine the following factors: accuracy of the 
final part, material and mechanical properties, the time required to manufacture 
the part, the need for post-processing, the size of \gls{AM} machine, and the overall 
cost of the machine and the process \cite{gibson_additive_2014}. 
Initially, \gls{AM} was used to manufacture prototypes. 
However, with improvements in material properties, accuracy, and overall quality 
of \gls{AM} output, the applications for \gls{AM} expanded to the 
point at which some industries build parts for direct assembly purposes
\cite{uriondo_present_2015}.  
Furthermore, using \gls{AM} in conjunction with other technologies, such as 
high-power lasers, has enabled \gls{AM} to manufacture parts made from various 
metals \cite{gibson_additive_2014}. 

\subsection{Application of Additive Manufacturing to Nuclear Reactor Core Components}
\label{sec:am}
\gls{AM} has progressed rapidly in the last 30 years, from rapid design prototyping 
with polymers in the automotive industry to scale production of metal components.  
Examples include Boeing using \gls{AM} to reduce the 979 Dreamliner's weight 
\cite{noauthor_printed_2017} and General Electric using \gls{AM} to produce fuel 
injection nozzles \cite{noauthor_transformation_2018}. 
The most common metal \gls{AM} technologies, \gls{SLM}, \gls{EBM}, \gls{L-DED}, 
and binder jetting, are not currently used to manufacture nuclear power plant parts. 
Wide-spread adoption of these methods in the nuclear industry could drastically 
reduce fabrication costs and timelines, combine multiple systems and assembled 
components into single parts, increase safety and performance by tailoring 
local material properties, and enable geometry redesign for optimal load paths 
\cite{simpson_considerations_2019}. 
Many Generation IV advanced reactor concepts have complex geometries, 
such as hex-ducts for sodium-cooled fast reactors, that are costly and difficult 
to fabricate using standard processing techniques. 
Traditional manufacturing routes also restrict many viable geometries for 
reactor designers to explore \cite{sridharan_performance_2019}. 
In summary, the main benefits of using \gls{AM} for reactor core components is that 
we are no longer geometrically constrained by conventional fuel manufacturing 
and can further optimize and improve fuel geometries to enhance fuel performance 
at lower costs \cite{bergeron_early_2018}. 

Experimental work in the nuclear materials' field demonstrates the application 
of \gls{AM} to nuclear fuel and structural core material fabrication. 
Rosales et al. \cite{rosales_characterizing_2019} conducted a feasibility study 
of direct routes to fabricate dense uranium silicide (U$_3$Si$_2$) fuel pellets 
using the \gls{INL} invented \gls{AMAFT}. 
U$_3$Si$_2$ is an accident-tolerant nuclear fuel candidate due to its
high uranium density and improved thermal properties. 
Its current metallurgical fabrication process is expensive and long; the goal of
\gls{AMAFT} is to fabricate U$_3$Si$_2$ at a lower cost in a timely and
commercially-reliable manner \cite{rosales_characterizing_2019}.  
Sridharan et al. \cite{sridharan_performance_2019} demonstrated the application of
the laser-blown-powder \gls{AM} process to fabricate \gls{FM} steel, a type of 
steel commonly used for cladding and structural components in nuclear reactors. 
Koyanagi et al. \cite{koyanagi_additive_2020} presented the latest 
\gls{AM} technology for manufacturing nuclear-grade \gls{SiC} materials; they 
demonstrated that combinations of \gls{AM} techniques and traditional \gls{SiC} 
densification methods enabled new designs of \gls{SiC} components with complex shapes. 
\gls{SiC} has excellent strength at elevated temperatures, chemical inertness, 
relatively low neutron absorption, and stability under neutron irradiation up 
to high doses \cite{sauder_ceramic_2014, snead_handbook_2007,koyanagi_additive_2020}. 
These qualities make \gls{SiC} suitable for many applications in nuclear systems 
such as fuel cladding, constituents of fuel particles \cite{snead_handbook_2007} 
and pellets \cite{terrani_progress_2015}, and core structural components in fission 
reactors \cite{sauder_ceramic_2014}. 
Trammel et al \cite{trammell_advanced_2019} conducted a preliminary investigation 
to assess the possibilities of fabricating a fuel element with embedded 
\gls{TRISO} fuel using \gls{AM} techniques, such as binderjet printing and \gls{CVI}. 
They successfully demonstrated a fabrication method using the following steps 
(depicted in Figure \ref{fig:ornl-triso-print}): 
\begin{enumerate}
    \item A SiC fuel element structure is first printed with binderjet technology. 
    \item The designated fueled region of the element is loaded with surrogate 
    \gls{TRISO} particles and additional SiC powder to fill interstitial spaces
    between particles. 
    \item The loaded fuel element is densified in a \gls{CVI} process to achieve 
    microencapsulation of \gls{TRISO} particles in a SiC matrix. 
\end{enumerate}
\begin{figure}[]
    \centering
    \includegraphics[width=\linewidth]{ornl-triso-print.png} 
    \caption{Stages of \gls{AM} fabrication conducted at \acrlong{ORNL} to 
    produce a fuel element with non-homogenously shaped coolant channels and
    \gls{TRISO} particles embedded in a SiC matrix \cite{trammell_advanced_2019}.}
    \label{fig:ornl-triso-print}
\end{figure}

Many of the materials and fabrication methods discussed are applicable 
for \gls{FHR}-part manufacturing. 
Therefore, this reiterates the possibility of leveraging \gls{AM} to 3D print a
\gls{FHR}-type reactor with non-conventional geometry. 

\section{Nuclear Reactor Design Optimization}
\label{sec:opt}
The practice of nuclear reactor optimization has been around since the 
conception of nuclear reactors. 
Optimization has been applied to many nuclear engineering sub-fields such 
as reactor design, reactor reloading patterns, and the nuclear fuel cycle.  
In the proposed work, we will focus on the reexamination of nuclear reactor core 
design optimization for arbitrary reactor geometries and fuel distributions. 
Previous efforts towards nuclear reactor core design optimization were limited by 
traditional manufacturing constraints and utilized deterministic and stochastic 
optimization techniques, coupled with surrogate models. 

Deterministic optimization methods usually start from a guess solution.
Then, the algorithm suggests a search direction by applying local 
information to a pre-specified transition rule. 
The best solution becomes the new solution, and the above procedure continues 
several times \cite{deb_multi-objective_2001}. 
Drawbacks of deterministic methods include: algorithms tend to get stuck at a 
suboptimal solution, and an algorithm efficient in solving one type of problem 
may not solve a different problem efficiently \cite{deb_multi-objective_2001}. 
Stochastic optimization methods, such as evolutionary algorithms amd simulated annealing,  
minimize or maximize an objective function when randomness is present; they 
tend to find globally optimal solutions more reliably than deterministic methods. 

A nuclear reactor's complexity results in reactor design optimization being a 
multi-objective design problem requiring a tradeoff between desirable 
attributes \cite{byrne_evolving_2014,simon_sciences_2019}. 
When multiple conflicting objectives are important, there is no single optimum 
solution that simultaneously optimizes all objectives. 
Instead, the multi-objective optimization problem's outcome is a set of optimal 
solutions with varying degree objective values \cite{deb_multi-objective_2001}. 
For a multi-objective problem like reactor design optimization, 
an ideal multi-objective optimization method should find widely spread solutions 
in the obtained non-dominated front \cite{deb_multi-objective_2001}. 

Recent efforts towards nuclear reactor optimization have relied heavily on 
the aforementioned stochastic methods, with the occasional addition of 
stochastic-deterministic hybrid methods. 
Sacco et al. \cite{sacco_two_2006,sacco_metropolis_2008} used stochastic 
simulated annealing and deterministic-stochastic hybrid optimization techniques 
to optimize reactor dimensions, enrichment, materials, etc., in order to 
minimize the average peak factor in a three-enrichment-zone reactor. 
Odeh et al. \cite{odeh_core_2016} used the simulated annealing stochastic algorithm 
coupled with neutronics and thermal-hydraulics codes, \gls{PARCS} and RELAP5
\cite{fletcher_relap5mod3_1992}, to develop an optimum \gls{NMR-50} core design 
to achieve a 10-year cycle length with minimal fissile loading. 
Kropaczek et al. \cite{kropaczek_large-scale_2019} demonstrated the constraint 
annealing method: a highly scalable method based on the method of parallel 
simulated annealing with mixing of states \cite{kropaczek_constraint_2019} for 
the solution of large-scale, multiconstrained problems in \gls{LWR} fuel cycle 
optimization. 
Peireira et al. \cite{pereira_coarse-grained_2003,pereira_parallel_2008} 
used a coarse-grained parallel genetic algorithm and a niching genetic algorithm
to optimize the same problem as Sacco et al. \cite{sacco_two_2006}. 
Kamalpour et al. \cite{kamalpour_smart_2020} utilized the imperialist competitive 
algorithm, a type of evolutionary algorithm, to optimize an \gls{FCM} fuelled 
\gls{PWR} to extend the reactor core cycle length. 

Nuclear reactor optimization problems require computationally 
extensive neutronics and thermal-hydraulics software to compute the objective 
function and constraints. 
Multiple papers utilized stochastic optimization methods with surrogate models 
to replace computationally expensive, high-fidelity neutronics or thermal hydraulics 
simulations to reduce computational cost.
Kumar et al. \cite{kumar_new_2015} combined genetic algorithm optimization 
with a surrogate model to optimize for high breeding of $^{233}$U and $^{239}$Pu 
in desired power peaking limits and keff by varying these parameters: fuel pin 
radius,  fissile material isotopic enrichment, coolant mass flow rate, and 
core inlet coolant temperature.
Betzler et al. \cite{betzler_design_2019} developed a systematic approach to 
build a surrogate model to serve in place of high-fidelity computational 
analyses. 
They leveraged the surrogate model with a simulated annealing optimization 
algorithm to generate optimized designs at a lower computational cost to 
understand design decisions' impact on desired metrics for \gls{HFIR} \gls{LEU} 
core designs.

The \gls{SA} method uses a point-by-point approach:
one solution gets updated to a new solution in one iteration, which does not 
exploit parallel systems' advantages.
Finding an optimal solution with \gls{SA} methods takes very 
long if high-fidelity computationally expensive codes are used to compute 
the objective function and constraints.
Therefore, using the \gls{SA} method is only practical if a 
surrogate evaluation model is used as described in Betzler et al. 
\cite{betzler_design_2019} and Kumar et al. \cite{kumar_new_2015}.

Contrary to a single solution per iteration in deterministic and stochastic 
\gls{SA} methods, evolutionary algorithms use a population of solutions in each 
iteration \cite{deb_multi-objective_2001}. 
evolutionary algorithm methods mimic nature's evolutionary principles to drive 
its search towards an optimal solution. 
With the affordability and availability of parallel computing systems, the 
evolutionary algorithm optimization method stands out as a method 
that easily and conveniently exploits parallel systems. 
Further, evolutionary algorithms have proved amenable to \gls{HPC} solutions and 
scalable to tens of thousands of processors \cite{kropaczek_constraint_2019}. 
Therefore, in this proposed work, we will utilize the evolutionary algorithm 
optimization method. 

\subsection{Impact of Additive Manufacturing on Nuclear Reactor Design 
Optimization}
Previous efforts towards nuclear reactor optimization, as discussed in the above 
section, focused on optimizing classical reactor parameters such as 
radius of fuel pellet and clad, enrichment of fuel, pin pitch, etc. 
With the advancements of \gls{AM} for reactor core components, reactor designers 
are no longer geometrically constrained by conventional fuel manufacturing and 
can optimize beyond classical parameters to enhance fuel performance and safety 
further. 
Reactor design objectives remain consistent with past objectives, such as 
minimizing fuel amount and minimizing the maximum fuel temperature for a given 
power level.
However, we can now approach the nuclear design problems with truly arbitrary 
geometries, no longer limited by traditional geometric shapes that are 
easy to manufacture with traditional processes: slabs as fuel planks, cylinders 
as fuel rods, spheres as fuel pebbles, axis-aligned coolant channels, etc  
\cite{sobes_artificial_2020}.
This has opened the door for a re-examination of reactor core 
optimization in a complete new way, determining the optimal arbitrary geometry 
for a given objective function \cite{sobes_artificial_2020} with a much smaller 
set of constraints. 

With a substantial increase and change in an arbitrary geometry's design space, 
it becomes time consuming for a human reactor designer to thoroughly explore 
and find optimal geometries in the expanded design space. 
Instead, we can leverage \gls{AI} optimization methods (such as evolutionary algorithms) to 
promptly explore the large design space to find global optimal designs. 
\gls{AI} would not replace the human reactor designer but shifts the human 
designer's focus away from conjecturing suitable geometries to defining design 
criteria to find optimal designs \cite{sobes_artificial_2020}. 
Thus, when the human designer changes the reactor criteria, the \gls{AI} 
model will quickly adapt and produce new global optimal designs to fit the new 
criteria.  

\section{Evolutionary Algorithms} 
\gls{AI} research is the study of `intelligent agents': any device that perceives 
its environment and takes actions that maximize its chance of successfully 
achieving its goals \cite{david_l_poole_computational_1998}.
Therefore, evolutionary algorithms are considered a subset of \gls{AI} as they create a population 
of individual solutions, inspired by biological evolution, and induce goals by 
using a `fitness function' to mutate and preferentially replicate high-scoring 
individuals to reach an optimal solution.
Evolutionary algorithms often perform well at approximating solutions to many 
problem types because they do not make any assumptions about the 
underlying fitness landscape.
The most popular evolutionary algorithms used to solve multi-objective problems are genetic algorithms 
\cite{byrne_evolving_2014, krish_practical_2011}. 

\subsection{Genetic Algorithms}
\label{sec:genetic_alg}
Genetic algorithms imitate natural genetics and selection to evolve solutions 
by maintaining a population of solutions, allowing fitter solutions to reproduce
and letting lesser fit solutions die off, resulting in final solutions that are 
better than the previous generations \cite{renner_genetic_2003}. 
From here, we will refer to a solution as an individual within the population. 
genetic algorithms efficiently exploit historical information to speculate new search 
points, improving each subsequent population's performance 
\cite{goldberg_genetic_1989}. 
Genetic algorithms are theoretically and empirically proven to provide robust 
search in complex spaces and are computationally simple yet powerful 
in their search for improvement \cite{goldberg_genetic_1989}. 
GAs are advantageous compared to deterministic and stochastic simulated 
annealing optimization methods, because (1) it searches from a population of 
points; (2) uses objective function information, not derivatives or other 
auxiliary knowledge of the problem; and (3) uses probabilistic transition 
rules, not deterministic rules. 
Figure \ref{fig:genetic_alg} depicts the iterative process of using a genetic algorithm
to solve a problem. 
The genetic algorithm generates new populations iteratively until it meets the termination 
criteria. 

\begin{figure}[]
        \centering
        \begin{tikzpicture}[node distance=1.7cm]
                \tikzstyle{every node}=[font=\small]
                \node (1) [lbblock] {\textbf{Create initial population}};
                \node (2) [lbblock, below of=1] {\textbf{Evaluate initial population}};
                \node (3) [lbblock, below of=2, yshift = -1.3cm] {\textbf{Create new population:} \\ 
                \begin{enumerate} \item \textbf{Select} individuals for mating 
                                  \item Create offspring by \textbf{crossover} 
                                  \item \textbf{Mutate} selected individuals 
                                  \item Keep selected individuals from previous generation
                                 \end{enumerate}};
                \node (4) [lbblock, below of=3, yshift=-1.3cm] {\textbf{Evaluate new population}};
                \node (5) [lbblock, below of=4] {\textbf{Is termination criteria satisfied?}};
                \node (6) [lbblock, below of=5] {\textbf{Best solution is returned!}};
                \draw [arrow] (1) -- (2);
                \draw [arrow] (2) -- (3);
                \draw [arrow] (3) -- (4);
                \draw [arrow] (4) -- (5);
                \draw [arrow] (5) -- node[anchor=east] {yes} (6);
                \draw [arrow] (5) -- ([shift={(0.5cm,0cm)}]5.east)-- node[anchor=west] {no} ([shift={(0.5cm,0cm)}]3.east)--(3);
        \end{tikzpicture}
        \caption{Process of finding optimal solutions for a problem with a 
        evolutionary algorithm \cite{renner_genetic_2003}. }
        \label{fig:genetic_alg}
\end{figure}

Genetic algorithms use mechanisms inspired by biological evolution such as selection, 
crossover, and mutation. 
The three operators are simple and straightforward. 
The selection operator selects good individuals. 
The crossover operator recombines good individuals to form a better 
individual. 
The mutation operator alters individuals to create better individuals
\cite{deb_multi-objective_2001}. 
Next, we provide more description and common methods for each operator.

\subsubsection{Selection Operator}
The selection operator's primary objective is to duplicate good 
individuals and eliminate bad individuals while keeping the population 
constant \cite{deb_multi-objective_2001}. 
It achieves this by identifying above-average individuals in a population, 
eliminating bad individuals from the population, and replacing them with 
copies of good individuals.
Selection operator methods utilized in the proposed work include tournament 
selection, best selection, and NSGA-II selection. 
In tournament selection, tournaments are played between a user-defined number 
of individuals, and the best individual is kept in the population. 
This repeatedly occurs until all the population's spots are filled. 
In best selection, a user-defined number of best individuals are selected, 
and copies are made to keep the population size constant. 
In NSGA-II selection, parent and offspring populations are combined, and the
best individuals (with respect to fitness and spread) are selected
\cite{deb_fast_2002}.
Again, copies of the best individuals are made to keep the population size constant. 

The selection operator cannot create any new individuals in the population 
and only makes more copies of good individuals at the expense of not-so-good
individuals. 
Instead, crossover and mutation operators perform the creation of new solutions.

\subsubsection{Crossover Operator}
The crossover operator is also known as the mating operator. 
In most crossover operators, two individuals are picked from the population at 
random, and some portion of the individuals' attributes are exchanged with one 
another to create two new individuals \cite{deb_multi-objective_2001}. 
Crossover operator methods utilized in the proposed work include single-point
crossover, uniform crossover, and blend crossover. 
In the single-point crossover, two individuals are selected from the population,
and a site along the individual's definition is randomly chosen. 
Attributes on this cross site's right side are exchanged between the two 
individuals, creating two new offspring individuals.  
In a uniform crossover, the user defines an independent probability for each 
individual's attribute to be exchanged; usually, $p=0.5$ is used. 
In blend crossover, two offspring (O) individuals are created based on a linear 
combination of two-parent (P) individuals using the following equations: 

\begin{align}
    O_1 &= P_1 - \alpha(P_1-P_2) \\
    O_2 &= P_2 + \alpha(P_1-P_2)
\intertext{where}
\alpha &= \mbox{Extent of the interval in which the new values can be drawn} \nonumber \\
 & \mbox{for each attribute on both side of the parents’ attributes (user-defined)} \nonumber 
\end{align}

To preserve some good individuals selected during the selection 
operator stage, not all individuals are used in a crossover; this is implemented 
by having the user define a crossover probability ($p_c$).  
Therefore, the crossover operator is only applied to $100p_c\%$ of the 
population; the rest are copied to the new population
\cite{deb_multi-objective_2001}. 

The crossover operator is mainly responsible for the search aspect of the 
genetic algorithms, whereas the mutation operator is needed to keep diversity 
in the population \cite{deb_multi-objective_2001}. 

\subsubsection{Mutation Operator}
The mutation operator alters one or more attributes of an individual within 
a population. 
Mutation occurs in the genetic algorithm based on a user-defined mutation probability 
($p_m$) that is set low to prevent a primitive random search. 
Mutation operator methods utilized in the proposed work include polynomial 
bounded mutation, in which each attribute in each individual is mutated 
based on a polynomial distribution. 
The user also defines each attribute's upper and lower bounds and the 
crowding-degree of the mutation, $\eta$ (a large $\eta$ will produce a mutant 
resembling its parent, while a small $\eta$ will produce the opposite).  

\subsection{Balancing Genetic Algorithm Hyperparameters}
In the proposed work, hyperparameters refer to parameters whose value controls 
the genetic algorithm's process, such as each genetic operator's associated parameters 
and population size.  
A well-performing genetic algorithm needs to balance the extent of exploration and 
exploitation; this is done by finding a balance between the conservation of 
valuable individuals obtained until the current generation while exploring new 
individuals. 
If previously obtained individuals are exploited too much, the population loses 
its diversity, and premature convergence to a suboptimal solution is expected. 
Alternatively, if too much stress is given on exploration, the information obtained 
thus far has not been appropriately utilized, and the genetic algorithm's search procedure 
behaves like a random search process \cite{deb_multi-objective_2001}. 
A quantitative balance between these two issues, exploitation and exploration, 
is challenging to achieve. 
Deb et al. \cite{deb_multi-objective_2001} and Goldberg et al. 
\cite{goldberg_toward_1993} quantified the relationship between exploitation 
and exploration. 
They found that for the one-max test problem, in which the objective is to 
maximize the number of 1s in a string, a genetic algorithm with any arbitrary 
hyperparameter setting does not work well even on a simple problem. 
Only genetic algorithms with a selection pressure (s) and crossover probability ($p_c$) 
falling inside the control map (Figure \ref{fig:controlmap}) will find the desired 
optimum.  
\begin{figure}[]
    \centering
    \includegraphics[width=0.6\linewidth]{controlmap.png} 
    \caption{Control map region of selection pressure (s) and crossover probability ($p_c$)
    values in which the genetic algorithm will find the desired optimum for the 
    one-max problem \cite{goldberg_toward_1993,deb_multi-objective_2001}.}
    \label{fig:controlmap}
\end{figure}
Another consideration is the population size. 
A function with considerable variability in objective function values demands 
a large population size to find a global optimum \cite{deb_multi-objective_2001}. 

Therefore, finding an optimized solution with genetic algorithms require the user 
to conduct a hyperparameter search. 
Ng et al. \cite{ng_improving_2021} suggest that a coarse to fine sampling scheme 
is the best way to perform a systematic hyperparameter search.  
For a two-dimensional example of a coarse to fine sampling scheme, the user 
first does a coarse sample of the entire square. 
A fine search is then conducted on the best-performing region of the coarse 
search. 

\section{Summary}
This chapter provided a literature review of relevant past research 
efforts that give context to this proposed work.
In summary, additive manufacturing of nuclear reactor components is a quickly 
developing field thanks to the aerospace and auto industries, which led to 
breakthroughs in \gls{AM} fabrication of metal components. 
The promise of cheaper and faster manufacturing of reactor components with 
\gls{AM} frees complex reactor geometries from previous manufacturing constraints
and allows reactor designers to reexamine reactor design optimization.  
Stochastic optimization methods such as evolutionary algorithms have proven to 
work well for finding global optimums in multi-objective design problems such as 
nuclear reactor optimization and can be leveraged to explore the vast exploration 
design space enabled by \gls{AM}.